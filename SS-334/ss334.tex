\documentclass{article}

\usepackage[english]{babel}

\usepackage[letterpaper,top=2.5cm,bottom=2.5cm,left=2.5cm,right=2.5cm,marginparwidth=1.75cm]{geometry}

\usepackage{amsmath}
\usepackage{graphicx}
\usepackage[colorlinks=true, allcolors=blue]{hyperref}

\title{SS-334: Microeconomics}
\author{Jacob Sigman}
\date{}

\begin{document}
\maketitle
\section*{Chapter 1: An Introduction to Economics}
\begin{description}
    \item [Definition of Economics] All economic questions arise because we want more than we can get. Our inability to satisfy all our wants is called \textbf{scarcity}. Because we face scarcity, we must make \textbf{choices}. The choices we make depend on the incentives we face. An \textbf{incentive} is a reward that encourages an action or a penalty that discourages an action.
    \begin{itemize}
        \item \textbf{Economics} is the social science that studies the choices that individuals, businesses, governments, and entire societies make as they cope with scarcity and the incentives that influence and reconcile those choices. It's divided into microeconomics and macroeconomics.
        \item \textbf{Microeconomics} is the study of choices that indivisuals and businesses make, the way those choices interact in markets, and the influence of governments.
        \item \textbf{Macroeconomics} is the study of the performance of the national and global economics.
    \end{itemize}
    \item [Two Big Economic Questions] Two big questions summarize the scope of economics. The first is: how do choices end up determining what, how, and for whom goods and services get produced? The second is: when do choices made in the pursuit of self-interest also promote the social interest? \textbf{Goods and services} are the objects that people value and produce to satisfy human wants. goods and services are produced by using productive resources that economists call \textbf{factors of production}, which are grouped into four categories: land, labor, capital, and entrepeneurship.
    \begin{itemize}
        \item The gifts of nature that we use to produce goods and services are \textbf{land}.
        \item The work time and work effort that people devote to producing goods and services is \textbf{labor}.
        \item The quality of labor depends on \textbf{human capital}, which is the knowledge and skill that people obtain from education, on-the-job training, and work esperience.
        \item The tools, instruments, machines, buildings, and other constructions that businesses use to produce goods and services are \textbf{capital}.
        \item The human resource that organizes land, labor, and capital is \textbf{entrepeneurship}.
        \item Land earns \textbf{rent}, labor earns \textbf{wages}, capital earns \textbf{interest}, entrepeneurship earns \textbf{profit}.
        \item You make choices that are in your \textbf{self-interest}, choices that you think are  best for you.
        \item Choices that are best for society as a whole are said to be in the \textbf{social interest}. This has two dimensions: efficiency and fair shares.
        \item Resource use is \textbf{efficient} if it is not possible to make someone better off without makins someone else worse off. The idea that social interest requires fair shares is a deeply held one.
        \item \textbf{Globalization} is the expansion of international trade, borrowing, lengind, and investment.
    \end{itemize}
\end{description}
\section*{Chapter 2: The Economic Problem}
\begin{description}
    \item [Production Possibilities and Opportunity Cost] The \textbf{production possibilities frontier (PPF)} is the boundary between those combinations of goods and services that can be produced and those that cannot. Every choice along the PPF involves a \textbf{tradeoff}.
    \item [Gains from Trade] A person has an \textbf{absolute advantage} if that person is more productive than others. A person has a \textbf{comparative advantage} in an activity if that person can perform the activity at a lower opportunity cost than anyone else. Absolute advantage involves comparing productivities while comparative advantage involves comparing opportunity costs.
    \item [Economic Growth] The expansion of production possibilities and an increase in the standard of living is called \textbf{economic growth}. To use resources in research and development and to produce new capital, we must decrease our production of consumption goods and services. The opportunity cost of economic growth is less current consumption.
    \begin{itemize}
        \item \textbf{Technological change} is the development of new goods and of better ways of producing goods and services.
        \item \textbf{Capital accumulation} is the growth of capital resources, which includes human capital.
    \end{itemize}
    \item [Economic Coordination] The choices of two actors, firms and households, and of two markets must be coordinated: firms, households, markets for factors of production, and markets for goods and services. Factors of production and goods and services flow in one direction while money flows in the opposite direction. Markets coordinate individual decisions through price adjustments.
\end{description}
\section*{Chapter 3: Supply and Demand}
\begin{description}
    \item [Markets and Prices] A \textbf{market} is any arrangement that enables buyers and sellers to get information and do business with each other. A \textbf{competitive market} is a market that has many buyers and many sellers so no single buyer or seller can influence the price. The \textbf{money price} of a good is the amount of money needed to buy it. The \textbf{relative price} of a good, the ratio of its money price to the money price of the next best alternative good, is its opportunity cost.
    \item [Demand] The \textbf{law of demand} states that the higher the price, the lower the demand and the lower the price, the higher the demand. The term \textbf{demand} refers to the entire relationship between the price of the good and the quantity demanded of the good. A \textbf{demand curve} shows the relationship between the quantity demanded and its price.
    \begin{itemize}
        \item When some influence on buying plans other than the price of the good changes, there is a \textbf{change in demand} for that good.
        \item When the demand increases the demand curve shifts to the right. When the demand decreases, the demand curve shifts to the left.
        \item Six main factors change demand: prices of related goods, expected future prices, income, expected future income and credit, population, and preferences.
        \item A \textbf{substitute} is a good that can be used in place of another good.
        \item A \textbf{complement} is a good that is used in conjunction with another good.
        \item When income increases, consumers buy more and the demand curve shifts to the right. A \textbf{normal good} is one for which demand increases as income increases. An \textbf{inferior good} is a good for which demand decreases as income increases.
        \item The greater the population, the greater the demand.
    \end{itemize}
    \item [Supply] If a firm supplies a good or service, then the firm has the resources and technology to produce it, can profit from producing it, and has made a definite plan to produce and sell it. \textbf{Resources} and \textbf{technology} determine what is posssible to produce. The \textbf{quantity supplied} of a good or service is the amount that producers plan to sell during a given time period at a particular price. The \textbf{law of supply} states that the higher the price of a good, the higher the supply, and the lower the price, the lower the supply. The term \textbf{supply} refers to the relationship between the quantity supplied and its price. A \textbf{supply curve} shows the relationship between the quantity supplied and its price.
    \begin{itemize}
        \item When some influence on selling plans other than the price of the good changes, there is a \textbf{change in supply} of that good.
        \item When the supply increases, the supply curve shifts to the right. When the supply decreases, the supply curve shifts to the left.
        \item Six main factors change supply: prices of factors of production, prices of related goods, expected future prices, the number of suppliers, technology, and the state of nature.
    \end{itemize}
\end{description}
\section*{Chapter 6: Government Price Controls: Price Ceilings, Price Floors, and Taxes}
\begin{description}
    \item [Market Equilibrium] Equilibrium is a situation in which opposing forces balance each other. Equilibrium in a market occurs when the price balances the plans of buyers and sellers. The \textbf{equilibrium price} is the price at which the quantity demanded is the same as the quantity supplied. The \textbf{equilibrium quantity} is the quantity bought and sold at the equilibrium price.
    \begin{itemize}
        \item A \textbf{surplus} is when supply is greater than demand.
        \item A \textbf{shortage} is when demand is greater than supply.
    \end{itemize}
    \item [Government Price Controls] The government sometimes intervenes by dictating prices. A \textbf{price ceiling} determines the maximum price. A \textbf{price floor} is a regulation that makes it illegal to trade at a price lower than a specific level. If it's \textbf{non-binding}, it has no effect. If it's \textbf{binding}, it has an effect.
    \item [Test] test.
\end{description}
\end{document}
