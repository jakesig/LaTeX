\documentclass{article}

\usepackage[english]{babel}
\usepackage[letterpaper,top=2.5cm,bottom=2.5cm,left=2.5cm,right=2.5cm,marginparwidth=1.75cm]{geometry}
\usepackage{amsmath, graphicx, tikz, pgfplots, multirow, newlfont, gensymb, indentfirst}

\usepackage{fancyhdr}
\pagestyle{fancy}
\fancyhf{}
\rhead{Midterm Practice}
\lhead{CE-344\\Environmental Systems Engineering}
\cfoot{\thepage}
\renewcommand{\headrulewidth}{1.5pt}
\setlength{\headheight}{22.6pt}
\usepackage[colorlinks=true, allcolors=black]{hyperref}
\setlength\parindent{24pt}

\begin{document}
\section*{Closed Book}
\begin{enumerate}
    \item List the three factors that affect the saturation solubility of oxygen in water.\\
    \rule{5cm}{1pt}\\
    Temperature, forces, bonds, and pressure.
    %\vspace{20mm}
    \item Define the hardness of water. In what units do we measure it?\\
    \rule{5cm}{1pt}\\
    Hardness in water is the measure of Calcium and Magnesium in it. It's measured in milliequivilents per liter.
    \item What is COD of a wastewater? In what units do we measure it, and is it going to be larger or smaller (numerically) than the BOD value for the same wastewater?\\
    \rule{5cm}{1pt}\\
    The COD of a wastewater is the Chemical Oxygen Demand and is measured in mg/L. It will always be higher than BOD. 
    \item With a sketch and a brief description, explain how the fall overturn occurs in deep lakes and reservoirs of the NYC Tri-State area.\\
    \rule{5cm}{1pt}\\
    \item Sketch and label the parts of the DO Sag Curve of an open microbial system in river water.\\
    \rule{5cm}{1pt}\\
    \item List four factors that affect the ``allocated per person'' water use in a community.\\
    \rule{5cm}{1pt}\\
    \item With regards to their use of free molecular oxygen, how many types (name them) of bacteria do we have? Do the ones who do not use it, use oxygen or not, and where do they get it from (if they do)?\\
    \rule{5cm}{1pt}\\
    \item List four places we use manholes in a wastewater collection system.\\
    \rule{5cm}{1pt}\\
    Changes of pipe size, changes in sewer grade, changes in alignment, and at the end of each line.
    \item List four factors contributing to the Eutrophication of a lake.\\
    \rule{5cm}{1pt}\\
    Nutrient enrichment, hydrodynamics, temperature, salinity, and biodiversity.
    \item Define alkalinity of a water. What is it's purpose in natural water bodies?\\
    \rule{5cm}{1pt}\\
    Alkalinity is the measure of water to neutralize acids. It's purpose in water is to determine the proper dosages of treatment chemicals.
    \item With a sketch and labels, show the degradation conditions (zones) of a lake due to an inflow of wastewater and the resulting DO Sag Curve.\\
    \rule{5cm}{1pt}\\
    \item List four reasons we locate a manhole in a sewer network.\\
    \rule{5cm}{1pt}\\
    Pipe inspection, pipe cleaning, periodic flushing.
    \item List the factors that influence the maximum solubility of oxygen (saturation value) in a lake.\\
    \rule{5cm}{1pt}\\
    \item Which are two methods used for coming up with a numerical estimate of \emph{E. Coli} (indicator) bacteria in water/wastewater samples? Briefly outline the main steps of the process.\\
    \rule{5cm}{1pt}\\
    Multiple-Tube Fermentation Technique and Presence-Absence Technique.
    \item Why do we design sewer networks as branched systems, and why do we consider a minimum flow for each pipe's design size?\\
    \rule{5cm}{1pt}\\
\end{enumerate}
\end{document}