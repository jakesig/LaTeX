\documentclass{article}

\usepackage[english]{babel}
\usepackage[letterpaper,top=2.5cm,bottom=2.5cm,left=2.5cm,right=2.5cm,marginparwidth=1.75cm]{geometry}
\usepackage{amsmath, graphicx, tikz, pgfplots, multirow, newlfont, gensymb, indentfirst, bm, setspace, xurl}
\usepackage[usestackEOL]{stackengine}
\usepackage{fancyhdr}
\pagestyle{fancy}
\fancyhf{}
\rhead{Jacob Sigman\\11/28/22}
\lhead{CE-344\\Environmental Systems Engineering}
\cfoot{\thepage}
\renewcommand{\headrulewidth}{1.5pt}
\setlength{\headheight}{22.6pt}
\usepackage[colorlinks=true, allcolors=black]{hyperref}
\setlength\parindent{24pt}

\begin{document}
    \begin{titlepage}
    \begin{center}
    {{\Large{\textsc{The Cooper Union for the Advancement of Science and Art}}}} \rule[0.1cm]{15.8cm}{0.1mm}
    \rule[0.5cm]{15.8cm}{0.6mm}
    {\small{\bf DEPARTMENT OF CIVIL AND ENVIRONMENTAL ENGINEERING}}\\
    {\footnotesize{STRUCTURAL ENGINEERING LABORATORY}}
    \end{center}
    \vspace{15mm}
    \begin{center}
    {\large{\bf LAB 5\\}}
    \vspace{5mm}
    {\Large{\bf COMPRESSIVE TESTING OF\\}}
    \vspace{2mm}
    {\Large{\bf STANDARD CONCRETE CYLINDER}}
    \end{center}
    \vspace{35mm}
    \par
    \noindent
    \hfill
    \vspace{20mm}
    \begin{center}
    {\large{ {\bf Group 2} \\ { Jenna Manfredi\hspace{5mm}David Madrigal\hspace{5mm}Gila Rosenzweig\\Nicole Shamayev\hspace{5mm}Jake Sigman}}}
    \vspace{40mm}
    {\large {\bf \\CE-321 \\ 12/14/22 \\}}
    \vspace{15mm}
    {\normalsize{Professor Tzavelis \\ Avery Kugler \\ Lionel Gilliar-Schoenenberger \\ Crystal Woo}}
    \end{center}
\end{titlepage}
    \begin{spacing}{2.15}
    \tableofcontents
    \newpage
    \section{General Electric Digital Wind Farm}
    \indent The Digital Wind Farm and General Electric uses digital technology that is used to address a long-needed need for greater flexibility in renewable power. General Electric is the leader in the transformation of the wind power industry with the world's first ``Digital Wind Farm''. The many turbines along with the digital infrastructure allows for enhanced production, lower costs, and higher operating efficiency. Once the turbines are built, General Electric uses sensors to collect data and analyze them in real time. Temperatures, vibrations, and anything else that can affect the performance of the turbines are recorded. This technology is installed in 4,000 units. Turbine efficiency has improved as high as 5\%, translating to up to a 20\% increase in profitability for each turbine.\\\\
    \url{https://www.ge.com/news/press-releases/ge-launches-next-evolution-wind-energy-making-renewables-more-efficient-economic}
    \section{Tata Power Solar Rooftop Panels}
    \indent The agency for Non-Conventional Energy along with the Government of India have joined to harness the immense power of the sun using the technologies of Tata solar power. Over 7700 systems were ordered from Tata Power Solar, which is the largest order for a single vendor. Each solar array was custom designed due to the diversity in roofing types (flat, tiled, sloping, etc.). This had an incredibly large impact on many families. The power generated from these panels suffice the energy needs of families which stress the power grid in India much less. Solar systems for seniors are crucial in India. Most seniors don't have basic electric appliances such as lights, fans, and televisions. They additionally have gained access to communication devices.\\\\
    \url{https://www.tatapowersolar.com/project/10-8-mw-rooftop-solar-power-system-anert-kerala/}
    \newpage
    \section{Lean Six Sigma Methods at 3M}
    \indent 3M has been a leader in using Lean Six Sigma methods to improve operations. Lean Six Sigma is a process-improvement methodology and a collection of statistical tools that help to improve product quality and process efficiency. Overall, this program has helped 3M to reduce volatile air emissions by about 61\%, improve energy efficiency, reduce waste, and double the number of their 3P projects. 3M's 3P (Pollution Prevention Pays) projects have helped to prevent more that 2.6 billion pounds of pollutants. These projects primarily focus on product reformulation, equipment redesign, and finding new ways of recycling materials. These projects have grown to become reliant on the new Lean Six Sigma methodology with much more ambitious goals for the future.\\\\
    \url{https://www.epa.gov/sustainability/3m-lean-six-sigma-and-sustainability}
    \section{Power in Waves in South Africa}
    \indent It's no surprise that drinking water has become incredibly expensive in Africa. While the oceans are endless, desalination techniques have proven expensive and time-consuming. Resolute Marine Energy's technology allowed for a wave-driven desalination system. Overall, this system provides for cheaper water which uses renewable energy in the form of ocean waves. Waves carry a surprising amount of power, especially with water being over 800 times denser than air. These systems are easily scalable. They're packaged in standard marine containers and can be installed relatively quickly to meet the quickly growing needs of Africans. Overall, this helps to evade the growing need for redevelopment of pipelines and canals.\\\\
    \url{https://mitsloan.mit.edu/teaching-resources-library/resolute-marine-energy-power-waves}
    \newpage
    \section{Industrial Water Pollution in Southern China}
    \indent In China, water pollution is a serious problem. 12.62 billion tons of polluted materials and 8.3 billion tons of wastewater polluted the waters of Guangdong in 2007. The apparel industry suffered greatly from polluted water. However, BSR launched the China Water Initiative. The company identified a range of pressures of factory managers that impact wastewater management and wastewater discharge. They partnered with many organizations such as the Institute for Public and Environmental Affairs, and the China Environment Forum. Overall, BSR is teaching factory managers more efficient and environmentally sound ways of conducting textile mills and dyeing facilities. This has allowed for less wastewater, more efficient energy use, and efficiency in the use of chemicals and materials.\\\\
    \url{https://www.bsr.org/en/our-insights/case-study-view/cleaning-up-industrial-water-pollution-in-southern-china}
    \section{Coca Cola's Sustainability}
    \indent Coca Cola has made a promise to become more sustainable, reducing the carbon footprint of their business operations by 15\% by 2020. Their main facility in Brampton converted to an energy-efficient lighting system that not only uses 50\% less energy, but provides 50\% more light. They additionally operate on motion sensors for the most energy efficiency. In vending machines, energy management devices have been installed for smarter lighting and cooling, improving energy efficiency by up to 35\%. Additionally, a 20\% reduction in water use was achieved. Coca Cola took action that implemented recycle and reclaim water loops through their water treatment systems. They were able to reclaim nearly 11 million liters of water in production of their products.\\\\
    \url{https://www.environmentalleader.com/2011/08/coca-cola-a-case-study-in-sustainability/}
    \newpage
    \section{California Giant Berry Farms Sustainability Efforts}
    \indent On this farm, sustainability is very dedicated. This farm has a strong Zero Waste Certification initiative. This farm uses a lot of measures to ensure they become as sustainable as they can. The first measure they used is TRUE, or Total Resource Use and Efficiency. This is a whole-systems approach that alters how materials flow through society in order to ensure that there is no waste. They also implemented an Environmentally Preferred Purchasing Policy, which advises that priority should be given to materials that are sourced from suppliers that are also committed to zero waste. In addition to this, they generate e-waste, batteries, fluorescent bulbs, and even the occasional aerosol can. These are stored in a safe storage area to prevent pollution.\\\\
    \url{https://www.freshproduce.com/resources/sustainability/case-studies/california-giant-berry-farms-sustainability-case-study/}
    \section{Zespri Sustainable Packaging Solutions}
    \indent Zespri has altered the way they package their products. One of their first objectives was to ensure full visibility on their consumer packs. In doing this, they developed a Global Consumer Packaging Management System. This system allows insight into all the packaging types they use and gives them a substantially better understanding of the materials being used along with the carbon footprint of Zespri. This allowed for them to change their packaging types to allow for a much lower carbon footprint. Zespri works closely with Product Innovation Teams to manage the packaging system, strategically determining any customers and retailers they need to work with to make any crucial switches in packaging. Now, they're using reusable polyethylene terephthalate in their packaging.\\\\
    \url{https://www.freshproduce.com/resources/sustainability/case-studies/zespri-sustainability-case-study/}
    \newpage
    \section{Good Energy's Biodiverse Solar and Wind Farms}
    \indent Delabole was the United Kingdom's first commercial wind farm, with outstanding performance. Delabole is owned by a company called Good Energy. This company wanted to become entirely decentralized, giving communities a larger say in how they use their energy. The solar farms use species-rich grassland to enhance the existing pasture. Sheep will be allowed to graze on sites of wind farms to show that renewable energy and agriculture can thrive as one. The characteristics of Good Energy are much different from other renewable energy companies, as Good Energy promised to work with customers and businesses to find the best places for renewable energy sites. With immense growing support for renewable energy in the United Kingdom, Good Energy has become a big contender for the energy sector.\\\\
    \url{https://www.theguardian.com/sustainable-business/2015/apr/30/good-energy-makes-biodiverse-solar-and-wind-farms}
    \section{The Social Stock Exchange}
    \indent The Social Stock Exchange, or the SSX wants to limit investing to investments that make a lasting impact on the environment. This exchange targets both retail and institutional investors. The more investors that invest in the exchange, the more capital that becomes available for many businesses to increase their social and environmental impacts. However, this exchange is not regulated. Companies have to take part of an intense application process, including independent evaluations. According to the business development manager at SSX, only about 45\% of applicants actually are able to become a part of the exchange. One of the biggest challenges the SSX faces is to evaluate and measure the impact. Members would be required to file impact reports and assess their performance against standards.\\\\
    \url{https://www.theguardian.com/sustainable-business/2016/may/27/alternative-stock-exchange-promotes-profit-and-social-impact}
    \newpage
    \section{Irrigation App in Spain}
    \indent A company in Spain, Innocent Drinks, partnered with farmers to examine varieties of irrigation equipment based on soil types, climate conditions, and plant types. From these examinations, a new app arrived, Irri-fresa. This app calculates optimal irrigation times based on the aforementioned conditions. Farmers who have used this app have cut water use by as high as 40\%. Water is an important resource overall in Spain. The 1.7 billion liters saved by farmers contributes to various other initiatives in Spain. These initiatives include solar and wind power. Water is required to operate all of these renewable energy sources, and this app might be making a small contribution, but every little bit counts in contributing to the carbon footprint.\\\\
    \url{https://www.theguardian.com/sustainable-business/2016/may/27/innocent-drinks-app-helps-spain-strawberry-farmers-slash-water-use}
    \section{Sustainable Digital Flavor Taps Using AWS}
    \indent Britvic's Aqua Libra Flavor Tap can sit on any cabinet-sized unit. The design includes a touchscreen that allows for a user interface and the device dispenses a drink in seconds. This is powered by Amazon Web Service's Internet of Things. As Britvic tracked customer behavior by analyzing real-time data on the usage of the device, it was found that it was best for recyclable cups to be placed alongside the tap. As this device optimizes various aspects of the workplace, it also has strong effects on the environment, preserving water and energy while encouraging the healthy habit of drinking more throughout the day. It even works to determine optimal locations for the machines in the workplace based on how many customers are using them at each location.\\\\
    \url{https://aws.amazon.com/solutions/case-studies/britvic-case-study/}
    \newpage
    \section{Smart-Home Energy Management App}
    \indent For many, rooftop solar panels are not a realistic option. A company called Lightsource Labs created an app called Tribe, which allows for intelligent home energy management. The application runs on artificial intelligence, learning consumption patterns. Once the patterns are learned, it manages the consumption based on the price signals of the electricity market and weather information. All the user has to do is install a Tribe hub in their home. Overall, it's unrealistic for people to manage home energy consumption on their own. This app that analyzes and manages the energy consumption of homes throughout the country has contributed greatly to preserving the energy of homes worldwide.\\\\  
    \url{https://aws.amazon.com/solutions/case-studies/lightsource-bp-case-study/}
    \section{An Irrigation System with a Brain}
    \indent Nefatim is the world's largest irrigation company by market share. It has 17 plants that manufacture precise drip irrigation equipment, that gets sold in over 100 countries. Overall, the guiding principle of the firm is to help farmers grow more with less. As climate change develops, water preservation has become crucial in irrigation systems worldwide. Netafim has brought NETBEAT to the market, the world's first irrigation system with a brain. Sensors are placed in a field to collect data about crops. This information then gets sent to cloud-connected units that control the fertilizer and irrigation systems. This system also pulls information from the internet such as weather forecasts and smart algorithms for plant growth.\\\\
    \url{https://aws.amazon.com/solutions/case-studies/netafim-video/}
    \newpage
    \section{Lyft's Emissions Offset}
    \indent Emissions from the transportation sector have been a growing issue. 3Degrees worked with Lyft to design a program to offset emissions. Overall, Lyft's initial investment used carbon offsets as a tool to contribute to reductions in emission. Many projects in automotive manufacturing and waste oil recycling were short-term projects. Vehicle electrification was and continues to be a medium-term project. Clearly, it was found that many of these issues have a direct connection to the automotive supply chain. Lyft was able to fund many companies to act as a catalyst to implement changes that are more sustainable and clean. Lyft was able to significantly offset emissions and this program became amongst the top 10 emissions initiatives in the world.\\\\
    \url{https://3degreesinc.com/resources/lyft-combats-climate-change-with-every-ride/}
    \section{Removing Tannins from Wastewater}
    \indent The New York State Pollution Prevention Institute worked with a paper manufacturer to determine practical, cost-effective options to remove tannins from wastewater. Many experiments were performed including Dissolved Air Filtration. The test for Dissolved Air Filtration was unsatisfactory. It was determined that many other treatment steps would be required. However, nanofiltration membrane technology gave satisfactory water quality. Despite giving satisfactory water quality, it was very cost-prohibitive. It was determined that a settling tank filter bed system was the most cost-effective and efficient method of removing tannins from wastewater. This system has contributed substantially to wastewater treatment all over the country.\\\\
    \url{https://www.rit.edu/affiliate/nysp2i/sites/rit.edu.affiliate.nysp2i/files/docs/resources/Paper_Manufacturer_Identifies_Options_for_Removing_Tannins_in_Process_Water.pdf}
    \newpage
    \section{Patagonia's Environmental Program}
    \indent The apparel industry is responsible for as much as 7\% of greenhouse gas emissions globally. Patagonia has worked immensely to reduce their emissions. So far, with their sustainability projects, they've arrived at 100\% renewable energy for all their stores, offices, and distribution centers. They continue their strong efforts to conserve water and remove toxins everywhere possible. They have switched to many organic materials including organically grown cotton and virgin cotton. They launched a set of ``preferred materials''. These materials include hemp, recycled polyester, recycled nylon, and regenerative organic cotton. The company has additionally joined many organizations including the sustainable apparel coalition. This has helped immensely to lower their carbon footprint.\\\\
    \url{https://www.patagonia.com/our-responsibility-programs.html}
    \section{Reduction of Greenhouse Gas Emissions in Japan}
    \indent During the late 1940's, the Japanese economy was highly dependent on the production of coal as it's primary source of energy. In 1967, the Advisory Council for Energy began pursuing the goals of decreasing the sulphur content of heavy oil along with stabilizing petroleum supplies. Laws put in place for this implementation established pollution control measures in highly industrialized areas. The 1973 oil shock motivated Japan strongly to become less reliant on oil as it accounts for a total of 80\% of the energy demand in Japan. Energy conservation was promoted, energy sources were diversified, and alternative energy resources were found and developed. These measures have been able to substantially reduce emissions of greenhouse gases in Japan.\\\\
    \url{http://personal.colby.edu/personal/t/thtieten/ener-jap.html}
    \newpage
    \section{Wind Power in Denmark}
    \indent Investments in alternative energy sources have become incredibly important for companies that rely on fossil fuels for energy. Denmark has been pushing a strong initiative to implement diverse and renewable energy sources. The country has been able to maintain 20\% renewable energy. Additionally, Denmark has a much higher proportion of wind-generated power per person than any other country in the world. Denmark has grown to host half of the worlds largest wind turbine manufacturers. The top three manufacturers in Denmark produce over half of all turbines worldwide. Denmark's immense success in wind power has been unmatched. Their early entry into wind power created a substantial efficiency advantage and immense strides in sustainability and renewable energy. \\\\
    \url{http://personal.colby.edu/personal/t/thtieten/Vogel.htm}
    \section{ESPN Goes Green}
    \indent ESPN created a major project to reduce solid waste and conserve energy and all campuses and remote events. ESPN's main campus has a Green Team of 80 employees that meets twice every month. ESPN began tracking its waste diversion waste in 2004, which was 48\%. The company then performed a waste audit, closely analyzing where it was generating trash and how to reduce said generation. It was found that food waste was a large source. ESPN separated food waste and sent it to be composted. This allowed for ESPN to divert over 14 tons of waste each quarter. Additionally, one third of ESPNS's vehicles are electric. Many lighting initiatives have been made including new lighting choices and the implementation of motion sensors, and even solar power to supply 7\% of the required power. \\\\
    \url{https://portal.ct.gov/-/media/DEEP/p2/newsletter/P2ViewFall11pdf.pdf}
    \newpage
    \section{Anaerobic Digester on a Poultry Farm}
    \indent MAC Farms is a family farm in Kentucky. This farm operates eight chicken houses with about 21,000 broilers per house. Various efforts have been made to be more sustainable in chicken operations. One aspect is the implementation of anaerobic digestion systems. Bacteria is used to break down organic materials and can be used to produce electricity and heat. This process happens in anaerobic digesters. There are many sources for the anaerobic digesters. Primarily, a plethora of chicken litter gets generated from on-site houses. The digesters are useful because they can also accept materials from other locations, including manure, crops, and crop residue. These digesters have been successful in producing green energy and diverting landfill waste. \\\\
    \url{https://eec.ky.gov/Environmental-Protection/Compliance-Assistance/Spotlight%20Stories/ECAPSpotlightMACFarms2015.pdf}
    \section{Wastewater Management in Minnesota Correctional Facilities}
    \indent One particular project on a correctional facility in Minnesota called Lino Lakes was focused on determining waste streams throughout the facility and where the best opportunities were for waste reduction. It was found that organic waste generated in the kitchen department has been the highest contributor to landfill waste and the correctional facility. By sorting non-organic waste and organic waste, the recycling rate was able to increase by 30\% along with over 400,000 pounds of waste being diverted from landfills. Additionally, the standardization of recycling bins throughout the facility have made it easier for departments with diverse waste streams to properly sort wastes. This has increased the recycling rate by over 7\%. \\\\
    \url{https://nationalsbeap.org/files/nationalsbeap/Maggie-Kristian-Department-of-Corrections-Summary-2018.pdf}
    \newpage
    \section{EchoWorks Smart Sustainability}
    \indent EchoWorks is a social enterprise that recycles electronic waste. Electronic waste is one of the fastest growing segments of waste in the world. About 75\% of electronics continue to be stored in households due to the unavailability of convenient recycling options for electronics. Up to 70\% of heavy metals in landfills come from discarded electronics. EchoWorks has been instrumental in reducing the impact of electronic waste in landfills on the environment. In just the first three months of operation, about 1000 pounds of electronics were collected. By the time EchoWorks reached their third year of operation, they were able to recycle over 200,000 pounds of electronic waste just in Wisconsin. EchoWorks has been able to reduce the impact of electronic waste on the environment and contributes towards a more sustainable future. \\\\
    \url{https://hub.aashe.org/browse/casestudy/26315/EchoWorks-An-Electronics-Recycling-College-Community-Partnership}
    \section{Palo Alto College Community Garden}
    \indent It was found that 41\% of four-year students in Palo Alto are food insecure. The neighborhoods surrounding the college lack access to healthy, affordable food. The solution was to create an on-campus community garden to benefit both the student population in Palo Alto and the surrounding population. Over 500 pounds of produce was harvested and donated to food pantry on campus along with the San Antonio Food Bank. The garden was also able to combat nutrition-related health issues by encouraging the entire community to eat fresh produce and healthier food overall. The students learned the importance of water sources and that the implementation of a recycled water system was smart in implementing this community garden helping the world towards a more sustainable and healthy future. \\\\
    \url{https://hub.aashe.org/browse/casestudy/23395/Palo-Alto-College-Community-Garden}
    \newpage
    \section{Gold Tree Solar Farm}
    \indent The Gold Tree Solar Farm at California Polytechnic State University shows how campus buildings can be transformed. Overall, the solar farm has proven to be strongly effective, with projections of saving nearly \$17,000,000 over 20 years. On the site of the solar farm, the land is occupied by the animal science department to ensure continued sheep grazing and further research. Additionally, a new laboratory was created to monitor the entire grid of the farm. The farm has been able to provide nearly a quarter of the school's total electricity and has been able to offset nearly 1,500 metric tonnes of Carbon Dioxide emissions. Overall, this project served to demonstrate the effectiveness of implementing solar plants to produce renewable energy sources on college campuses. \\\\ 
    \url{https://hub.aashe.org/browse/casestudy/23268/Gold-Tree-Solar-Farm-Innovatively-Advancing-the-Academic-Mission}
    \section{The Good Store}
    \indent Food coordinators in British Columbia began to brainstorm ways for people to gain access to affordable, sustainable food sources. Soon enough, a plan was made to create a low-waste and sustainable bulk food store, with a strong emphasis on producing very little waste. This store was coined ``The Good Store''. The store has many pantry staples including snack foods and personal items. However, all of the items have minimal packaging. Customers are to bring their own bags to fill. If customers don't have their own bags, they are given glass jars free of charge. This allows for customers to shop without plastic. Many students on campus frequent the store. This store, while it's small-scale, demonstrates the importance of shopping with little waste. \\\\
    \url{https://hub.aashe.org/browse/casestudy/21844/The-Good-Store}
    \newpage
    \section{Sustainability in the Children's National Medical Center}
    \indent The Children's National Medical Center is not only a medical center; it's a sustainability hub. A sustainability project sparked so many aspects of this center. Every week, there's a farmers market along with a 2,500 square-foot vegetable garden. They have additionally implemented a single-stream recycling program to preserve resources. Additionally, to ensure greater sustainability, the center ensures that only post-consumer recycled paper and sustainable ink and toner are purchased. Water bottle filling stations are placed all around the facility to reduce the dependency on plastic water bottles commonly retrieved from vending machines. The center has a dedicated pharmaceutical waste program. Lastly, hydroculture plants beautify the space without any risk of bacteria or mold, improving the air quality in the center. \\\\
    \url{https://doee.dc.gov/page/childrens-national-medical-center-case-study-mayors-2013-sustainability-awards}
    \section{Water-Based Paints for Allston Collision Center}
    \indent Allston Collision Center was one of the first shops in Massachusets to switch to water-based paint. Water-based paints are paints that are prepared using water as opposed to solvent. For this auto repair shop, switching to water based paint eliminated over 1,200 pounds of VOCs (Volatile Organic Compounds) from yearly emissions. The center additionally implemented an efficient computer system that determines how much paint is required for each job. The result of using water-based paint is less overspray, meaning there are less fumes in the air as a result of the paint, and more paint on the car. Since the switch, only one car has ever required a solvent-based paint due to a specific color match. Even though water-based paint is more expensive, it lacks expense on the environment. \\\\ 
    \url{https://nationalsbeap.org/files/nationalsbeap/Case%2BStudy%2BAllston%2BCollision%2BCenter%2BOTA.2015%20%281%29.pdf}
    \newpage
    \section{Chemical Reduction in Wastewater Treatment}
    \indent A plant in Minneapolis processes about 270,000 gallons of treated wastewater per day. The current method being used is membrane cleaning and disinfection. Neutralized waste is sent to a holding pond and then to the Mississippi River. A more efficient way of Neutralizing the water would lower salt concentration and improve water quality. The implementation of self-neutralization was found to save an estimate of \$12,000 spent every year on wastewater treatment, without sacrificing any time in treating the water itself. Additionally, it was found that the recalibration of the pumps would immensely help to make the wastewater treatment process more efficient. Overall, chemical reductions in wastewater treatment has created more sustainable solutions for the treatment of wastewater. \\\\
    \url{https://nationalsbeap.org/files/nationalsbeap/Gina-Sternberg-City-of-Minneapolis-Water-Summary-2017.pdf}
    \section{Reduced Impact of Landscaping on Golf Courses}
    \indent A golf club in Maryland is creating sustainable efforts to run their course. Not only do they comply with any environmental regulation, but they created a project to reduce the impact of landscaping on the carbon footprint. Recycling containers have been installed at each of their 18 golf holes, which has been shown to reduce the municipal waste load by about 250 pounds per year. All mowing reels used in landscaping are electric. Additionally, every tree that must be removed for landscaping purposes results in the planting of two new trees of the same species at another location on the course. Lastly, the time of day fertilizers are applied increases absorption and decreases evaporation. This demands substantially less fertilizer. Creating sustainable ways to landscape a golf course has proven to substantially decrease the carbon footprint.\\\\
    \url{https://mde.maryland.gov/MarylandGreen/Documents/River-Downs-Golf-Club-Profile.pdf}
    \end{spacing}

\end{document}