\part*{Week 3}
\addcontentsline{toc}{part}{Week 3}
\section*{Chapter 3}
\underline{15-17, 20-23, 26}
\addcontentsline{toc}{section}{Chapter 3: 15-17, 20-23, 26}
\subsection*{Problem 15}
Compare the latency, persistence, and infective dose of \emph{Ascaris} and \emph{Salmonella}.
\\\rule{5cm}{1pt}
\begin{center}
    \renewcommand{\arraystretch}{1.2}
    \begin{tabular}{l l l l}
        \textbf{Bacteria} & \textbf{Latency} & \textbf{Persistence} & \textbf{Infective Dose} \\
        \hline
        \emph{Salmonella} & Latent & Persistent & High \\
        \emph{Ascaris} & Non-Latent & Kinda Persistent & Low
    \end{tabular}
\end{center}
\subsection*{Problem 16}
Historically in the United States, the prevalent infectious diseases were typhoid, cholera, and dysentery. How have these diseases been virtually eliminated? Currently, the prevalent infectious diseases are giardiasis and cryptosporidiosis, causing diarrhea that can be life-threatening for persons with immunodeficiency syndrome. What actions are being taken to reduce the probability of waterborne transmission of these diseases? (Refer to Section 5.)\\
\rule{5cm}{1pt}
\\\\Generally, human hygiene and overall sanitation have improved. The United States has been able to eliminate these diseases through the pasteurization of milk and the chlorination of water supplies. Giardiasis and cryptosporidiosis infect through poor water filtration. In order to mitigate the transmission, filtration systems must be properly chemically coagulated and chlorinated. 
\subsection*{Problem 17}
Discuss the significance of human carriers in transmission of enteric diseases. What major waterborne diseases in the United States are spread by carriers? How is the spread of two of these diseases amplified by animals?\\
\rule{5cm}{1pt}
\\\\Human carriers exist for all enteric diseases. As they carry these diseases, they may not exhibit symptoms, so infection can spread quickly. Human carriers are responsible for the continued transmission of the waterborne diseases \emph{Giardia lamblia} and \emph{Cryptosporidium}. These parasites also occur in domestic and feral animals. Animal-to-person contact may result in infection by \emph{Giardia lamblia}. Additionally, \emph{Cryptosporidium} may result in infection through animal feces.
\newpage
\subsection*{Problem 20}
In one statement, what is the general process in testing for \emph{Giardia} cysts and \emph{Cryptosporidium} oocysts? In method 1622, the water sample is only 10 L for testing natural stream water for \emph{Cryptosporidium} oocysts. Using this method to test stream samples at a variety of locations, why was the accuracy for detection and enumeration of oocysts low?\\
\rule{5cm}{1pt}
\\
\begin{enumerate}
    \item Filter the sample water for extraction.
    \item Remove the particulates from filter.
    \item Concentrate the particulates by centrifugation.
    \item Separate the ocysts from the debris.
    \item Stain with a fluorescent antibody for microscopic detection.
\end{enumerate}
The accuracy for detection and enumeration of ocysts were low in method 1662 because cross-reacting algae were present in water samples.
\subsection*{Problem 21}
Why must laboratories conducting tests for \emph{Cryptosporidium} oocysts be audited and approved for quality assurance?\\
\rule{5cm}{1pt}
\\\\There have been various methods developed by the EPA for testing for \emph{Cryptosporidium}, and \emph{Cryptosporidium} ocysts are identified by color, size, and shape. Since there is no safe way of treatment for \emph{Cryptosporidium} ocysts, it is important to avoid public health concerns.
\subsection*{Problem 22}
Why are coliform bacteria used as indicators of quality of drinking water? Under what circumstances is the reliability of coliform bacteria to indicate the presence of pathogens questioned?\\
\rule{5cm}{1pt}
\\\\Coliform bacteria are in the intestinal tract of humans and are excreted in many species of humans and warm-blooded animals. Water that is contaminated by feces is seen as dangerous by the presence of coliform bacteria. However, the presence of coliforms can only indicate wastes of humans, farm animals, or soil erosion. This means that the reliability of coliform bacteria to indicate the presence of pathogens declines substantially for any type of water other than drinking water.
\subsection*{Problem 23}
Coliform bacteria in surface waters can originate from feces of humans, wastes of farm animals, or soil erosion. Can the coliforms from these three different sources be distinguished from one another?\\
\rule{5cm}{1pt}
\\\\There is no way of distinguishing between the feces of humans and the wastes of farm animals. However, there is a special test that can be run to separate fecal coliforms from soil types.
\subsection*{Problem 26}
Why is lactose (milk sugar), an ingredient in all culture media, used to test for the coliform group?\\
\rule{5cm}{1pt}
\\\\Lactose is used in the process of fermenting Lactic Acid. Once the fermentation occurs, the multi-tube fermentation technique is able to enumerate total coliforms in wastewater and surface water.