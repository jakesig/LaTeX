\part*{Week 3}
\section*{Chapter 3}
\underline{15-17, 20-23, 26}
\subsection*{Problem 15}
Compare the latency, persistence, and infective dose of \emph{Ascaris} and \emph{Salmonella}.
\subsection*{Problem 16}
Historically in the United States, the prevalent infectious diseases were typhoid, cholera, and dysentery. How have these diseases been virtually eliminated? Currently, the prevalent infectious diseases are giardiasis and cryptosporidiosis, causing diarrhea that can be life-threatening for persons with immunodeficiency syndrome. What actions are being taken to reduce the probability of waterborne transmission of these diseases? (Refer to Section 5.)
\subsection*{Problem 17}
Discuss the significance of human carriers in transmission of enteric diseases. What major waterborne diseases in the United States are spread by carriers? How is the spread of two of these diseases amplified by animals?
\subsection*{Problem 20}
In one statement, what is the general process in testing for \emph{Giardia} cysts and \emph{Cryptosporidium} oocysts? In method 1622, the water sample is only 10 L for testing natural stream water for \emph{Cryptosporidium} oocysts. Using this method to test stream samples at a variety of locations, why was the accuracy for detection and enumeration of oocysts low?
\subsection*{Problem 21}
Why must laboratories conducting tests for \emph{Cryptosporidium} oocysts be audited and approved for quality assurance?
\subsection*{Problem 22}
Why are coliform bacteria used as indicators of quality of drinking water? Under what circumstances is the reliability of coliform bacteria to indicate the presence of pathogens questioned?
\subsection*{Problem 23}
Coliform bacteria in surface waters can originate from feces of humans, wastes of farm animals, or soil erosion. Can the coliforms from these three different sources be distinguished from one another?
\subsection*{Problem 26}
Why is lactose (milk sugar), an ingredient in all culture media, used to test for the coliform group?