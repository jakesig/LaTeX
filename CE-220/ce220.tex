\documentclass{article}

\usepackage[english]{babel}

\usepackage[letterpaper,top=2.5cm,bottom=2.5cm,left=2.5cm,right=2.5cm,marginparwidth=1.75cm]{geometry}

\usepackage{amsmath}
\usepackage{graphicx}
\usepackage[colorlinks=true, allcolors=blue]{hyperref}
\usepackage{cancel}
\usepackage{bm}

\title{CE-220: Fundamentals of Civil Engineering}
\author{Jacob Sigman}
\date{}

\begin{document}
\maketitle
\section*{Lecture 1 - 1/25/22}
\begin{itemize}
    \item Course Description
    \begin{itemize}
        \item Planning, execution, and interpretation of drawings and specifications for Civil Engineering projects.
        \item Sample drawings and specifications.
        \item Contractual requirements and sample contracts.
        \item Permitting, scheduling, and cost estimation.
        \item Basic operations of design and construction firms.
        \item Interface with other disciplines on Civil Engineering projects.
    \end{itemize}
    \item Midterm
    \begin{itemize}
        \item Likely March 8, before Spring Break.
        \item Multiple choice questions (might have multiple right answers)
    \end{itemize}
    \item Final group project/presentation
    \item Grading
    \begin{itemize}
        \item Class participation: 20\%
        \item Quizzes: 15\%
        \item HW: 20\%
        \item Midterm: 15\%
        \item Final Project: 30\%
    \end{itemize}
    \item Office Hours: 4:30 - 5:00, 8:00 - 8:30, by appointment
    \item 10 points deducted for each week that an assignment is late.
    \item Recommended readings: ENR, ASCE, any professional journals of interest
    \item Abbreviated notes will be posted in teams. Take notes like they won't be.
    \item Civil Engineering Sub-Disciplines
    \begin{itemize}
        \item Airport Engineering
        \item Architectural Engineering
        \item Coastal Engineering
        \item Construction Engineering
        \item Earthquake Engineering
        \item Environmental Engineering
        \item Forensic Engineering
        \item Geotechincal Engineering
        \item Highway Engineering
        \item Ports and Marine Engineering
        \item Materials Engineering
        \item Municipal/Urban Engineering
        \item Railway Engineering
        \item Site Engineering
        \item Structural Engineering
        \item Transportation Engineering
        \item Wastewater/Water Resources Engineering
    \end{itemize}
    \item Civil Engineers fulfill society's needs, a service profession.
    \item Introduction
    \begin{itemize}
        \item \textbf{The Process} - from Request for Qualifications and Proposal for initial Planning to Opening Day for the Project.
        \begin{itemize}
            \item Where it begins
            \item A ``Need'' is identified
            \begin{itemize}
                \item Owner needs to develop property purchased to lease for income (return on investment)
                \item Inspectors note that deck deterioration is advanced and needs repair/replacement.
                \item Trafic demands have grown to regularly ``jam'' the route and no viable alternates are available.
            \end{itemize}
            \item Scope developed - usually by owner or owner's representative (program manager for major projects)
            \item Request for Qualifications (RFQ) or Request for Proposal (RFP) for Design issued by Owners
            \begin{itemize}
                \item Lists \textit{qualifications} needed - (Sometimes 2-step process: RFQ first and shortlisted teams get the RFP second).
                \item Objectives and Scope of Work are detailed
                \item Schedule is defined
                \item Criteria
            \end{itemize}
        \end{itemize}
        \item \textbf{The Players} - Relationships among Owners, Designers, Builders (and sometimes Financers)
            \begin{itemize}
                \item Owner/Owner's Representative
                \item Designer/Engineer - Develops construction (or contract) Documents (CDs). Supports construction (reviews of Contractor's alternatives, RFIs, Means and Methods, relays design intent).
                \item Contractor - Bids on work defined in CDs. Lowest qualified bidder (usually) gets awarded the contract.
                \item Resident/Construction Inspector - Assures work is performed in accordace with CDs. Processes pay requisitions. Coordinates submissions to/from designer.
                \item Quality Control/Quality Assurance/Testing
                \item \textbf{Design-Bid-Build} Contractual relationships between owner and engineer and owner and contractor. Cooperative support between engineer and contractor.
                \item Roles civil engineers play: Designer, Resident/Owner's Representative, Contractor, Owner, Maintenance Engineer, QA/QC.
            \end{itemize}
        \item \textbf{New Construction} - Case Study - Tacoma Narrows Bridge
            \begin{itemize}
                \item Timeline for Tacoma Narrows Bridge
                \begin{itemize}
                    \item 1994 - WSDOT Public - Private Initiative Announced
                    \item 1996 - Major Investment Study
                    \item 1996/98 - Environmental Impact Studies
                    \item 1999 - Project Standards and Criteria Development
                    \item 2000 - Basic Configuration and Initial Design
                    \item 2001 - Determination of Fixed Price
                    \item 2002 - Legislation enacted and bonds shortlisted
                    \item 2002 - Notice to proceed - 9/25/2002
                    \item 2007 - Opening day - 7/17/2007
                \end{itemize}
                \item Financial mechanisms for procuring and paying for projects.
                \begin{itemize}
                    \item Buidlings v. Bridges
                    \item Procurement Methods
                    \item Conventional Design-Bid-Build (DBB)
                    \item Design/Build (DB) and Progressive Design/Build (PDB)
                    \item Public-Private Partnerships (P3) and Design-Build-Bid-Operate-Maintain (DBOM)
                    \item Construction Manager/General Contractor (CM/GC)
                    \item Last three are called alternate delivery (AltD)
                    \item Conventional Design-Bid-Build: Owner \(\rightarrow\) Design \(\rightarrow\) contract bid then built
                    \\ \textbf{Engineering Oriented}: Owner controlled, low risk, low opportunity.
                    \item Design-Build and P3: Owner \(\rightarrow\) 30-40\% Design and RFP \(\rightarrow\) Design/Build teams advance design, bid then final designed/built staged. Also adds finance/operate/maintain in P3.
                    \\ \textbf{Construction Oriented}: Contractor controlled, managed risk, better opportunity.
                    \item Progressive Design-Build: Owner \(\rightarrow\) 5-10\% Design and RFP \(\rightarrow\) PDB teams selected on qualifications, advance design with owner and owner's representative.
                    \\ \textbf{Investor Oriented}: Investor controlled, high risk, high opportunity.
                    \item CMGC - Owner ``brokers'' the marriage
                    \item Private public Partnerships, Design/Build/Operate/Maintain and other concepts
                    \item Bonding/Tolling and it's place in financing
                    \item Federally funded projects - interstate system
                    \item Real estate and tax implications
                \end{itemize}
                \item Contracts for Design
                \begin{itemize}
                    \item General Terms and Conditions: Standard of care, Insurance, Payment terms, other ``legalese''
                    \item Scope of Work
                    \item Compensation - types of Contracts
                    \item Schedule for project
                    \item Special provisions
                \end{itemize}
                \item Construction Inspection and Construction Management
                \item Contracts for Contractors - General terms and conditions (Division 1). The rest is the construction documents (plans and specifications, usually done by the design engineer)
            \end{itemize}
        \item \textbf{Rehabilitation} - Case Study - Verrazzano Narrows Upper Level Deck Replacement
        \begin{itemize}
            \item First phase - Study and design brief
            \begin{itemize}
                \item Notice to proceed - 12/2003
                \item Two viable operations: steel orthotropic and concrete filled steel grid.
                \item Traffic studies to determine workable staging
                \item Utility survey to evaluate relocation
                \item Analyses to ``global'' impact of each alternative
                \item Final recommendations
                \item Two conceptual (10\%) designs
                \item Budgetary cost estimates
            \end{itemize}
            \item Second Phase - Designer
            \begin{itemize}
                \item Two main construction contracts (Part A: Utility Relocation and Part B: Deck Replacement)
                \item Two prototypes (Trinidad Lake asphalt pavement at throggs neck bridge and orthotropic deck for fabrication ``proof of concept'' and fatigue tests)
                \item Additional Wind Tunnel Testing
                \item Value Engineering
                \item Constructability review
                \item Final Design - VN-90A - December 2008
                \item Survey - How Dissimilar might the panels be?
            \end{itemize}
        \end{itemize}
        \item \textbf{Recent Trends}
        \begin{itemize}
            \item Sustainability - Going ``Green'' neeeds to be part of process early if it will be followed through to completion.
            \item Integrated Project Delivery/BIM
        \end{itemize}
    \end{itemize}
    \item Homework 0
    \begin{itemize}
        \item Do one random act of kindness
        \item You cannot personally benefit from this
        \item You must not tell anyone what it is
        \item If the person you did it for finds out, it doesn't count
    \end{itemize}
\end{itemize}
\section*{Lecture 2 - 2/1/22}
\begin{itemize}
    \item Project documentation
    \begin{itemize}
          \item Contract/"Boilerplate"
          \item Specifications
          \item Plans
          \item Engineer's estimate
    \end{itemize}
    \item Bridges vs. Buildings
    \item Interdisciplinary projects
    \item Conflicts and Contradictions - minimizing them
    \item Civil Engineering Sub-Disciplines
    \begin{itemize}
        \item Airport Engineering
        \begin{itemize}
            \item JFK Terminal 4
            \item LaGuardia Airport re-envisioned
        \end{itemize}
        \item Architectural Engineering
        \begin{itemize}
            \item San Francisco's Salesforce tower
            \item Atlanta's Mercedes Benz Stadium
            \item Major coordination with architects and other trades
            \item Customers demanding more ``bells and whistles'' and ``moving parts''
            \item Facade and structural glass specialties are becoming a ``thing''
        \end{itemize}
        \item Coastal Engineering
        \begin{itemize}
            \item \underline{Waterfront work:} Levees and flood protection, bulkheads, seawalls, scour protection
            \item \underline{Beach erosion mitigation:} Jetties, groins, sand replenishment, delta preservation
            \item \underline{Offshore structures:} Oil rigs, wind farms, bridge pier scour protection, wave and tidal generators
        \end{itemize}
        \item Construction Engineering
        \begin{itemize}
            \item \underline{Means and Methods:} Staging, formwork and falsework, concrete curing plans.
            \item Traffic control plans
            \item Shop drawings
            \item Fabrication procedures
            \item Erection procedures
            \item \underline{Sizing cranes for construction:} Temporary track driven cranes, tower cranes, gantries
            \item Transportation problem solver
            \item Procurement
        \end{itemize}
        \item Earthquake Engineering
        \begin{itemize}
            \item Seismology
            \item Soil effects
            \begin{itemize}
                \item Soil-structure interaction - coming up with the springs
                \item Attenuation/amplification from intervening soil layers
            \end{itemize}
            \item Tectonics - movement of plates
            \item Monitoring and prediction modeling
            \item Stuctural analysis
            \begin{itemize}
                \item Response spectra
                \item Multi-modal response spectra
                \item Time history
                \item Design elements
            \end{itemize}
            \item Research and Development
            \begin{itemize}
                \item Active/passive damping systems
                \item Innovative bearings (isolation, friction/pendulum)
            \end{itemize}
        \end{itemize}
        \item Environmental Engineering
        \begin{itemize}
            \item Water treatment
            \begin{itemize}
                \item Controls for effluent and runoff
                \item Groundwater, settlement basins
                \item Desalination
                \item Waste treatment
                \item Cleanup from spills
            \end{itemize}
            \item Air
            \begin{itemize}
                \item Air quality effluent control/scrubbers, etc.
                \item Indoor air Quality: Dust control / filtering, Cleaning chemical pollutants
            \end{itemize}
            \item Noise control, indoor and out
            \item Soil-structure: Erosion controls (planting, hay bales, riprap, etc.)
        \end{itemize}
        \item Forensic Engineering
        \item Geotechincal Engineering
        \begin{itemize}
            \item Soils
            \begin{itemize}
                \item Geotechnical investigations
                \item Classifying
                \item Soil improvement: Stone columns, Soilcreting/jet grouting
                \item Settlement control/preconsolidation
                \item Highway and utility work
            \end{itemize}
            \item Foundations
            \begin{itemize}
                \item Spread footings and mats
                \item Pile foundations - many kinds
                \item Support of excavation: Tie-backs and sheeting, soil nailing
                \item Tunnels, shafts
            \end{itemize}
        \end{itemize}
        \item Highway Engineering
        \begin{itemize}
            \item Alignments - plan and profile
            \item Cross sections
            \item Mass haul optimization
            \item Utility plans and relocations
            \item Pavement boxes
            \item Curbs, sidewalks, paths, driveways
            \item Survey coordination
            \item Drainage
            \item Signalization and lighting
            \item Speed and red-light enforcement
            \item Traffic control
            \item Striping
            \item Signing
        \end{itemize}
        \item Ports and Marine Engineering
        \item Materials Engineering
        \begin{itemize}
            \item Metallurgy and alloying
            \begin{itemize}
                \item High performance steels
                \item Other metals
                \item Corrosion protection - Coatings, metalizing, cathodic protection, etc.
            \end{itemize}
            \item Concrete
            \begin{itemize}
                \item New mixes and materials: Glassphalt, lightweight aggregates, cements, fly ash, slag, and pozzolans, fiber reinforcement, ultra-high-performance concrete (UHPC)
            \end{itemize}
            \item Asphalts and binders
            \item Research and Development
            \begin{itemize}
                \item Plastics and composites, including carbon
                \item Fiber reinforced polymers (FRP)
                \item Nanotechnology
                \item Reuse of waste materials in new work
            \end{itemize}
        \end{itemize}
        \item Municipal/Urban Engineering
        \begin{itemize}
            \item Utilities/Infrastructure
            \begin{itemize}
                \item Communications - phone, broadband, cable, cell cites
                \item Electric and power generation/distribution
                \item Gas storage and distribution
                \item Steam
                \item Sewage/sanitary
                \item Pump stations
            \end{itemize}
            \item Mapping
            \item Geographical Information Systems (GIS)
            \item Parkland development and maintenance
            \item Streetscape
            \item Zoning and city Planning
            \item Maintenance
        \end{itemize}
        \item Railway Engineering
        \begin{itemize}
            \item Railroad design
            \begin{itemize}
                \item Alignments: Plan, Profile, tolerances get tighter with increased Speed
                \item Track work: Rail (continuously welded), frogs and switches, ballast, clamps
                \item Signals
                \item Platforms and ``gaps''
                \item Mezzanines and station design
                \item Bridge and tunnel design
                \item Embankments and retaining walls
            \end{itemize}
        \end{itemize}
        \item Site Engineering
        \begin{itemize}
            \item Permitting
            \item Site plans
            \item Drainage
            \item Sanitary sewers
            \item Parking lots
            \item Survey coordination
            \item Curbs, sidewalks, paths, driveways
            \item Utility plans and relocations
            \item Site and facility lighting
            \item Signing and striping
        \end{itemize}
        \item Structural Engineering
        \begin{itemize}
            \item Buildings
            \item Bridges
            \item Retaining walls
            \item Tunnels
            \item Special structures
            \begin{itemize}
                \item Guyed towers
                \item Blast design
                \item Shells and domes
                \item Fabric structures
                \item Stadiums
                \item Oil rigs
                \item Wind farms
                \item Transfer stations
                \item Ports and marine structures
            \end{itemize}
        \end{itemize}
        \item Transportation Engineering
        \begin{itemize}
            \item Transportation surveys
            \item Planning, modeling, and studies
            \item Operations
            \item Highway Systems
            \begin{itemize}
                \item Traffic projections
                \item Toll studies and financing
                \item Tolling methods
                \item Bike lanes and pedestrian paths
            \end{itemize}
            \item Mass transit
            \begin{itemize}
                \item Bus Systems
                \item Metro and light rail systems
                \item Commuter rail systems
                \item High-speed Rail
                \item Fare collection systems
            \end{itemize}
            \item Carpooling and other alternative transportation
        \end{itemize}
        \item Wastewater/Water Resources Engineering
        \begin{itemize}
            \item Water supply
            \item Testing and treatment
            \item Storage
            \item Distribution
            \item Pumping stations
            \item Maintenance
            \item Fire lines
            \item Desalination
            \item Wells and Aquifiers
            \item Irrigation
            \item Hydraulic Studies
            \begin{itemize}
                \item Dams
                \item River backwater studies
                \item Flooding studies
            \end{itemize}
        \end{itemize}
    \end{itemize}
    \item Sub-Disciplines: Wrap-up
    \begin{itemize}
        \item Lots to choose from
        \item Many overlap
        \item None are stagnant - continuous developments keep things interesting
        \item Plenty of long-term opportunities
    \end{itemize}
    \item Planning and permitting - Subject overview
    \begin{itemize}
        \item \textit{Private v. Public}: Who's in charge?
        \begin{itemize}
            \item Architects tend to take lead on private work/buildings
            \begin{itemize}
                \item Contract with the owners
                \item Subcontract to structural, mechanical, electical, and plumbing designers
                \item Make decisions on overall configuration
                \item Tend to be the ``LEEDers'' for Sustainability decisions
            \end{itemize}
            \item Civil Engineers tend to take the lead on public works/bridges
            \begin{itemize}
                \item Many major bridges do not have architectural involvement
                \item Geotech, Architects, MEP subs to structural or civil
            \end{itemize}
        \end{itemize}
        \item \textit{Planning}: Site selection, preliminary bedgeting, feasibility studies
        \begin{itemize}
            \item Site selection
            \begin{itemize}
                \item Owner purchases property, often in consultation with designers
                \item Owner objectives = ``program''
                \item Proximity to utilities/transportation
                \item LEED of ENV SP criteria
            \end{itemize}
            \item Preliminary Budgeting
            \begin{itemize}
                \item Does it make economic sense?
                \item Cost/benefit analysis
                \item Financing - bonds or loans needed?
            \end{itemize}
            \item Feasibility studies
            \begin{itemize}
                \item Any fatal flaws in the plan?
                \item Work arounds possible?
            \end{itemize}
        \end{itemize}
        \item \textit{Environmental} Assessment - Impacts to consider
        \begin{itemize}
            \item Water Quality
            \begin{itemize}
                \item Additional runoff created?
                \item Settlement ponds or permeable areas
            \end{itemize}
            \item Air Quality
            \item Dust and noise control during construction
            \item Additional traffic generated?
            \item Wildlife affected?
            \item Parkland
            \item Open Spaces
            \item Cultural Resources
            \item Historical Resources
            \item Natural Resources
            \item Quality of Life
        \end{itemize}
        \item Major investment studies
        \begin{itemize}
            \item Will it pay off?
            \item Depends on: Cost/benefit analysis, Life cycle costs, return on investment
            \item Financing options
        \end{itemize}
    \end{itemize}
\end{itemize}
\section*{Lecture 3 - 2/8/22}
\begin{itemize}
    \item Procurement Mechanisms for Civil projects
    \begin{itemize}
        \item ``Conventional'' - ``Design-Bid-Build''
        \begin{itemize}
            \item Designer selected by qualifications and often cost, workload and/or politics.
            \item Design developed to 100\% complete
            \item Usually several submissions (40\%, 70\%, 95\%)
            \item Each submission consists of several steps
            \begin{itemize}
                \item Design development and submission
                \item Owner comments
                \item Comment resolution
            \end{itemize}
            \item 100\% contract documents get released to bid
            \item Bid phase includes
            \begin{itemize}
                \item Pre-bid meeting/walkthrough
                \item Q\&A/RFIs from Contractors
                \item Issue Addenda/Clarifications/Amplifications if needed
                \item Bid
                \item Bid review: Verify qualifications for each bidder, watch for irregularities and unbalanced bids.
                \item Contract awarded (signed and executed)
            \end{itemize}
            \item Generally contract goes to ``Lowest qualified bidder''.
            \item Takes longer to implement
            \item Owner controls each step
            \item Owner procures 10-30\% design bid package: Package has criteria, rules established, package typically also has indicative plans.
            \item Package released to design/build teams to bid
            \item Design/Build teams
            \begin{itemize}
                \item Value engineer/Innovate
                \item Advance design - only enough to bid
                \item Sometimes ``qualifications package'' first
                \item Always bid
                \item Usually low bid wins, but not always
                \item Quality points often count
                \item Award made
                \item Construction starts as final design proceeds
                \item Owner reviews work packages along the way
                \item Task force meetings for all stakeholders
            \end{itemize}
            \item Now ``Standard operating procedure'' for MTA agencies for projects > \$25M
        \end{itemize}
        \item Work packages
        \begin{itemize}
            \item WPB1: Suspension system
            \item WPB2: Suspended superstructure
            \item WPB3: Towers
            \item WPB4: Anchorages
            \item WPB5: Miscellaneous Appurtenances
            \item WPB8: Deck Finishes
            \item Project Specs: 100\% Project specifications
            \item CES-003: Temporary Grout Tube Supports
            \item Construction Photographs
            \item Design
        \end{itemize}
        \item Private Public partnerships, Design/Bid/Build/Operate/Maintain and other concepts
        \begin{itemize}
            \item Like design build, and then some!
            \item Rules all vary greatly
            \item Not always low bid!
            \begin{itemize}
                \item Qualifications
                \item Financial terms, solvency, and serviceability and handoff
                \item ``Value'' to the ultimate owner
            \end{itemize}
        \end{itemize}
        \item Bonding/Tolling/Rent
        \begin{itemize}
            \item Financing - government bonds
            \item Toll Levels/Rent established to pay them back
            \item Tolls/rent collections also need to pay for maintenance, capital improvements
        \end{itemize}
        \item Federally Funded Projects - Interstate System and Mega Projects
        \begin{itemize}
            \item Major public works ``Helped make us a great nation''
            \item Keeps jobs, helps the economy get/keep going during construction
            \item Keeps transportation lines efficient
            \item Bipartisan infrastructure law passed
            \item Some becoming toll roads, make money
        \end{itemize}
        \item Real estate market
        \begin{itemize}
            \item Fluctuations in price are natural
            \item 3WTC history lesson
        \end{itemize}
        \item Tax implications
        \begin{itemize}
            \item Incentives - Amazon's HQ2
            \item Then it fell apart.
        \end{itemize}
    \end{itemize}
    \item Insurance and Bonding capacity can limit project size
    \begin{itemize}
        \item Contractor's Insurance Required
        \begin{itemize}
            \item Comprehensive general liability and property damage insurance
            \item Worker's compensation
            \item Marine liability
            \item Other insurance for special risk cases
        \end{itemize}
        \item Owner's protective liability
        \item Wrap-up project insurance for mega-projects
        \begin{itemize}
            \item Covers all parties
            \item More efficient
            \item Less common lately
        \end{itemize}
        \item Payment and performance bond
        \begin{itemize}
            \item Surety company
            \item Ratings
            \item Bonding capacity
        \end{itemize}
    \end{itemize}
    \item Homework 1
    \begin{itemize}
        \item Pick a recent project that you want to learn more about. Research the following items and present as a short essay in \underline{Five numbered outline/paragraphs corresponding to the list below}, citing sources along the way and \underline{\textbf{bulletizing}} as much as possible.
        \begin{enumerate}
            \item The players: A) Owner, B) Designer(s), and C) Contractor
            \item Initial idea: A) Whose idea, B) when, C) why, D) what need filled, etc.
            \item Timeline: Studies, permits, etc., through completion
            \item Funding Sources/Mechanisms to pay for the facility
            \item Procurement mechanism for the owner to get it done (DBB, DBBOM, etc.)
        \end{enumerate}
        \item Along the way, explain why you believe these guiding decisions were made the way they were.
    \end{itemize}
    \item A Bit on Calculations
    \begin{itemize}
        \item You will be well-prepared by the cooper union
        \item Calculations form the basis of our designs
        \item They must conform to the design criteria
        \item The results must be reflected in the design details
        \item Computers are stupid, much calculation (units units units)
    \end{itemize}
    \item A bit on drafting
    \begin{itemize}
        \item A bit of AutoCAD
        \item Scale is important!
        \item Cutting sections and showing details
    \end{itemize}
    \item A bit on technical writing
    \begin{itemize}
        \item Each form of writing has its own style
        \item Larger firms sometimes have their own style guide
        \item Your bosses will want you to emulate them
    \end{itemize}
    \item Drawing preparation and organization
    \begin{itemize}
        \item Title sheet
        \item List of drawings
        \item General plan and elevation
        \item Scope of work
        \item General notes
        \item Material notes
        \item Notes along the way: References to pay items, perscriptive instructions on means/method, don't repeat anything in specs.
        \item Abbreviations
        \item Existing conditions/demolition
        \item Make sure pay items in specifications are clear and fully coordinated.
    \end{itemize}
    \item Plan preparation and organization
    \begin{itemize}
        \item Plans show arrangement, include a North arrow and jump to: Sections, elevations, part plans, details
        \item Use cross-references to other drawings, specifications
        \item Sections, elevations
        \item Everything must be set out to ``tell a story'' and build towards an understanding of the ``whole''
        \item Scales - don't over-mix!
        \item Tie it all together, make usre everything is consistent and coordinated.
        \item Most clients want electronic delivery from PDF to CADD native files
        \item Plan prep for public vs. private sector
    \end{itemize}
    \item Horizontal Alignments - Horizontal Curves an Stationing
    \begin{itemize}
        \item Roads and rails aren't straight
        \item Stationing is how distance is measured
        \item 1 Station is 100'
        \item Rails and highways gradually change from tangent to curve
    \end{itemize}
\end{itemize}
\section*{Lecture 4 - 2/15/22}
\begin{itemize}
    \item Sample Structural drawings
    \item Legend
\end{itemize}
\section*{Lecture 5 - 2/22/22}
\begin{itemize}
    \item HW 2 and HW 3 are going to happen concurrently.
    \item March 8 is the midterm
    \begin{itemize}
        \item Part 1: Multiple choice (multiple right answers)
        \item Part 2: Short essay
    \end{itemize}
    \item San Francisco - Oakland Bay Bridge
    \begin{itemize}
        \item Sequence
        \item Format
        \item Linework
        \item Font Sizes
        \item Scales
        \item Plan views with north arrows
    \end{itemize}
\end{itemize}
\end{document}
