\documentclass{article}

\usepackage[english]{babel}

\usepackage[letterpaper,top=2.5cm,bottom=2.5cm,left=2.5cm,right=2.5cm,marginparwidth=1.75cm]{geometry}

\usepackage{amsmath}
\usepackage{graphicx}
\usepackage[colorlinks=true, allcolors=blue]{hyperref}
\usepackage{cancel}
\usepackage{bm}

\title{CE-220: Fundamentals of Civil Engineering}
\author{Jacob Sigman}
\date{}

\begin{document}
\maketitle
\section*{Lecture 1 - 1/25/22}
\begin{itemize}
    \item Course Description
    \begin{itemize}
        \item Planning, execution, and interpretation of drawings and specifications for Civil Engineering projects.
        \item Sample drawings and specifications.
        \item Contractual requirements and sample contracts.
        \item Permitting, scheduling, and cost estimation.
        \item Basic operations of design and construction firms.
        \item Interface with other disciplines on Civil Engineering projects.
    \end{itemize}
    \item Midterm
    \begin{itemize}
        \item Likely March 8, before Spring Break.
        \item Multiple choice questions (might have multiple right answers)
    \end{itemize}
    \item Final group project/presentation
    \item Grading
    \begin{itemize}
        \item Class participation: 20\%
        \item Quizzes: 15\%
        \item HW: 20\%
        \item Midterm: 15\%
        \item Final Project: 30\%
    \end{itemize}
    \item Office Hours: 4:30 - 5:00, 8:00 - 8:30, by appointment
    \item 10 points deducted for each week that an assignment is late.
    \item Recommended readings: ENR, ASCE, any professional journals of interest
    \item Abbreviated notes will be posted in teams. Take notes like they won't be.
    \item Civil Engineering Sub-Disciplines
    \begin{itemize}
        \item Airport Engineering
        \item Architectural Engineering
        \item Coastal Engineering
        \item Construction Engineering
        \item Earthquake Engineering
        \item Environmental Engineering
        \item Forensic Engineering
        \item Geotechincal Engineering
        \item Highway Engineering
        \item Ports and Marine Engineering
        \item Materials Engineering
        \item Municipal/Urban Engineering
        \item Railway Engineering
        \item Site Engineering
        \item Structural Engineering
        \item Transportation Engineering
        \item Wastewater/Water Resources Engineering
    \end{itemize}
    \item Civil Engineers fulfill society's needs, a service profession.
    \item Introduction
    \begin{itemize}
        \item \textbf{The Process} - from Request for Qualifications and Proposal for initial Planning to Opening Day for the Project.
        \begin{itemize}
            \item Where it begins
            \item A ``Need'' is identified
            \begin{itemize}
                \item Owner needs to develop property purchased to lease for income (return on investment)
                \item Inspectors note that deck deterioration is advanced and needs repair/replacement.
                \item Trafic demands have grown to regularly ``jam'' the route and no viable alternates are available.
            \end{itemize}
            \item Scope developed - usually by owner or owner's representative (program manager for major projects)
            \item Request for Qualifications (RFQ) or Request for Proposal (RFP) for Design issued by Owners
            \begin{itemize}
                \item Lists \textit{qualifications} needed - (Sometimes 2-step process: RFQ first and shortlisted teams get the RFP second).
                \item Objectives and Scope of Work are detailed
                \item Schedule is defined
                \item Criteria
            \end{itemize}
        \end{itemize}
        \item \textbf{The Players} - Relationships among Owners, Designers, Builders (and sometimes Financers)
            \begin{itemize}
                \item Owner/Owner's Representative
                \item Designer/Engineer - Develops construction (or contract) Documents (CDs). Supports construction (reviews of Contractor's alternatives, RFIs, Means and Methods, relays design intent).
                \item Contractor - Bids on work defined in CDs. Lowest qualified bidder (usually) gets awarded the contract.
                \item Resident/Construction Inspector - Assures work is performed in accordace with CDs. Processes pay requisitions. Coordinates submissions to/from designer.
                \item Quality Control/Quality Assurance/Testing
                \item \textbf{Design-Bid-Build} Contractual relationships between owner and engineer and owner and contractor. Cooperative support between engineer and contractor.
                \item Roles civil engineers play: Designer, Resident/Owner's Representative, Contractor, Owner, Maintenance Engineer, QA/QC.
            \end{itemize}
        \item \textbf{New Construction} - Case Study - Tacoma Narrows Bridge
            \begin{itemize}
                \item Timeline for Tacoma Narrows Bridge
                \begin{itemize}
                    \item 1994 - WSDOT Public - Private Initiative Announced
                    \item 1996 - Major Investment Study
                    \item 1996/98 - Environmental Impact Studies
                    \item 1999 - Project Standards and Criteria Development
                    \item 2000 - Basic Configuration and Initial Design
                    \item 2001 - Determination of Fixed Price
                    \item 2002 - Legislation enacted and bonds shortlisted
                    \item 2002 - Notice to proceed - 9/25/2002
                    \item 2007 - Opening day - 7/17/2007
                \end{itemize}
                \item Financial mechanisms for procuring and paying for projects.
                \begin{itemize}
                    \item Buidlings v. Bridges
                    \item Procurement Methods
                    \begin{itemize}
                        \item Conventional Design-Bid-Build (DBB)
                        \item Design/Build (DB) and Progressive Design/Build (PDB)
                        \item Public-Private Partnerships (P3) and Design-Build-Bid-Operate-Maintain (DBOM)
                        \item Construction Manager/General Contractor (CM/GC)
                        \item Last three are called alternate delivery (AltD)
                        \item Conventional Design-Bid-Build: Owner \(\rightarrow\) Design \(\rightarrow\) contract bid then built
                        \\ \textbf{Engineering Oriented}: Owner controlled, low risk, low opportunity.
                        \item Design-Build and P3: Owner \(\rightarrow\) 30-40\% Design and RFP \(\rightarrow\) Design/Build teams advance design, bid then final designed/built staged. Also adds finance/operate/maintain in P3.
                        \\ \textbf{Construction Oriented}: Contractor controlled, managed risk, better opportunity.
                        \item Progressive Design-Build: Owner \(\rightarrow\) 5-10\% Design and RFP \(\rightarrow\) PDB teams selected on qualifications, advance design with owner and owner's representative.
                        \\ \textbf{Investor Oriented}: Investor controlled, high risk, high opportunity.
                    \end{itemize}
                    \item CMGC - Owner ``brokers'' the marriage
                    \item Private public Partnerships, Design/Build/Operate/Maintain and other concepts
                    \item Bonding/Tolling and it's place in financing
                    \item Federally funded projects - interstate system
                    \item Real estate and tax implications
                \end{itemize}
                \item Contracts for Design
                \begin{itemize}
                    \item General Terms and Conditions: Standard of care, Insurance, Payment terms, other ``legalese''
                    \item Scope of Work
                    \item Compensation - types of Contracts
                    \item Schedule for project
                    \item Special provisions
                \end{itemize}
                \item Construction Inspection and Construction Management
                \item Contracts for Contractors - General terms and conditions (Division 1). The rest is the construction documents (plans and specifications, usually done by the design engineer)
            \end{itemize}
        \item \textbf{Rehabilitation} - Case Study - Verrazzano Narrows Upper Level Deck Replacement
        \begin{itemize}
            \item First phase - Study and design brief
            \begin{itemize}
                \item Notice to proceed - 12/2003
                \item Two viable operations: steel orthotropic and concrete filled steel grid.
                \item Traffic studies to determine workable staging
                \item Utility survey to evaluate relocation
                \item Analyses to ``global'' impact of each alternative
                \item Final recommendations
                \item Two conceptual (10\%) designs
                \item Budgetary cost estimates
            \end{itemize}
            \item Second Phase - Designer
            \begin{itemize}
                \item Two main construction contracts (Part A: Utility Relocation and Part B: Deck Replacement)
                \item Two prototypes (Trinidad Lake asphalt pavement at throggs neck bridge and orthotropic deck for fabrication ``proof of concept'' and fatigue tests)
                \item Additional Wind Tunnel Testing
                \item Value Engineering
                \item Constructability review
                \item Final Design - VN-90A - December 2008
                \item Survey - How Dissimilar might the panels be?
            \end{itemize}
        \end{itemize}
        \item \textbf{Recent Trends}
        \begin{itemize}
            \item Sustainability - Going ``Green'' neeeds to be part of process early if it will be followed through to completion.
            \item Integrated Project Delivery/BIM
        \end{itemize}
    \end{itemize}
    \item Homework 0
    \begin{itemize}
        \item Do one random act of kindness
        \item You cannot personally benefit from this
        \item You must not tell anyone what it is
        \item If the person you did it for finds out, it doesn't count
    \end{itemize}
\end{itemize}
\end{document}
