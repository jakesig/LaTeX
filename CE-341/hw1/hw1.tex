\documentclass{article}

\usepackage[english]{babel}

\usepackage[letterpaper,top=2.5cm,bottom=2.5cm,left=2.5cm,right=2.5cm,marginparwidth=1.75cm]{geometry}

\usepackage{amsmath}
\usepackage{graphicx}
\usepackage[colorlinks=true, allcolors=blue]{hyperref}
\usepackage{cancel, fancyhdr, textcomp, subfig}
\usepackage{bm, pgfplots}
\usepackage[export]{adjustbox}

\renewcommand{\thesubfigure}{}
\captionsetup[subfigure]{labelformat=simple, labelsep=colon}
\pagestyle{fancy}
\fancyhf{}
\rhead{Jacob Sigman\\1/25/23}
\lhead{CE-341\\Homework}
\cfoot{\thepage}
\renewcommand\arraystretch{1.5}
\renewcommand{\headrulewidth}{1.5pt}
\setlength{\headheight}{22.6pt}
\begin{document}
    \section*{Chapter 1}
    \subsection*{Question 1.5-1}
    The first step is to calculate the cross-sectional area of the metal specimen:
    \[A=(0.55\text{ in})^2\times\frac{\pi}{4}=0.238 \text{ in}^2\]
    The force at which fracture occurred was at a load of 28,500 pounds. The ultimate tensile strength can be found by dividing the force by the cross-sectional area.
    \[\sigma_\text{Ult}=\frac{F}{A}=\frac{28500 \text{ lb}}{0.238 \text{ in}^2}=119958.1\text{ psi}=\boxed{119.95\text{ ksi}}\]
    The first step in determining the elongation as a percentage is to determine the elongation in inches. The difference in the two gage lengths is used.
    \[2.3 \text{ in}-2.03\text{ in}=0.27\text{ in}\] 
    Next, take the determined difference and divide by the initial gage length to obtain the elongation as a percentage.
    \[\frac{0.27\text{ in}}{2.03\text{ in}}=0.133=\boxed{13.3\%}\]
    The initial cross sectional area was found to be 0.238 \(\text{in}^2\) as determined earlier. The cross-sectional area measured following the tensile test is found as follows:
    \[A=(0.43\text{ in})^2\times\frac{\pi}{4}=0.145 \text{ in}^2\]
    The reduction in cross-sectional area in square inches is found as follows:
    \[0.238 \text{ in}^2-0.145 \text{ in}^2=0.093\text{ in}^2\]
    Next, take the determined difference and divide by the initial cross-sectional area to obtain the reduction as a percentage.
    \[\frac{0.093\text{ in}^2}{0.238 \text{ in}^2}=0.389=\boxed{38.9\%}\]
    \newpage
    \subsection*{Question 1.5-6}
    Using the given cross-sectional area of 0.2011 square inches and the given gage length of 2 inches, stress is calculated by dividing the load by the cross-sectional area and strain is calculated by dividing the elongation by the gage length. 
    \begin{center}
        \begin{tabular}{|cccc|} 
            \hline
            \textbf{Load (kips)} & \textbf{Elongation x 1000 (in)} & \textbf{Stress (ksi)} & \textbf{Strain}  \\ 
            \hline
            0                    & 0                               & 0                     & 0                \\
            1                    & 0.160                           & 4.97                  & 0.00008          \\
            2                    & 0.352                           & 9.95                  & 0.00018          \\
            3                    & 0.706                           & 14.92                 & 0.00035          \\
            4                    & 1.012                           & 19.89                 & 0.00051          \\
            5                    & 1.434                           & 24.86                 & 0.00072          \\
            6                    & 1.712                           & 29.84                 & 0.00086          \\
            7                    & 1.986                           & 34.81                 & 0.00099          \\
            8                    & 2.286                           & 39.78                 & 0.00114          \\
            9                    & 2.612                           & 44.75                 & 0.00131          \\
            10                   & 2.938                           & 49.73                 & 0.00147          \\
            11                   & 3.274                           & 54.70                 & 0.00164          \\
            12                   & 3.632                           & 59.67                 & 0.00182          \\
            13                   & 3.976                           & 64.64                 & 0.00199          \\
            14                   & 4.386                           & 69.62                 & 0.00219          \\
            15                   & 4.640                           & 74.59                 & 0.00232          \\
            16                   & 4.988                           & 79.56                 & 0.00249          \\
            17                   & 5.432                           & 84.54                 & 0.00272          \\
            18                   & 5.862                           & 89.51                 & 0.00293          \\
            19                   & 6.362                           & 94.48                 & 0.00318          \\
            20                   & 7.304                           & 99.45                 & 0.00365          \\
            21                   & 8.072                           & 104.43                & 0.00404          \\
            22                   & 9.044                           & 109.40                & 0.00452          \\
            23                   & 11.310                          & 114.37                & 0.00566          \\
            24                   & 14.120                          & 119.34                & 0.00706          \\
            25                   & 20.044                          & 124.32                & 0.01002          \\
            26                   & 29.106                          & 129.29                & 0.01455          \\
            \hline
        \end{tabular}
    \end{center}
    \newpage
    \noindent Below is the stress-strain curve for the given values in the table above. The linear portion ends approximately at the point where the strain is 0.00318 and the corresponding stress is 94.48 ksi.
    \begin{center}
        \pgfplotsset{width=10cm}
        \begin{tikzpicture}[baseline=(current bounding box.center)]
            \begin{axis}[
                %title={\textbf{Stress-Strain Curve}},
                xlabel={Strain},
                ylabel={Stress (ksi)},
                xmin=0, xmax=0.015,
                ymin=0, ymax=150,
                xtick={0,0.005,0.01,0.015},
                ytick={0,50,100,150},
                ymajorgrids=true,
                grid style=dashed,
                legend pos = north west,
                legend cell align={left},
                xticklabel style={
                    /pgf/number format/fixed,
                    /pgf/number format/precision=5
                },
                scaled x ticks=false
            ]
            
            \addplot[smooth, red] table [x=strain, y=stress] {1.csv};
            %\addplot[blue] table [x=strain, y=stress_th] {1.csv};
            %\addlegendentry{Experimental}
            %\addlegendentry{Theoretical}
            \end{axis}
        \end{tikzpicture}
    \end{center}
    The modulus of elasticity is calculated by approximating the slope between (0.00318, 94.48) and (0, 0).
    \[E=\frac{94.48}{0.00318}=\boxed{29701.46 \text{ ksi}}\]
    The proportional limit is estimated to be approximately \(\boxed{105\text{ ksi}}\) using the stress-strain curve. To utilize the 0.2\% offset method, a strain of 0.002 is selected and a line is plotted based on the modulus of elasticity. The stress-strain curve with the offset is below.
    \begin{center}
        \pgfplotsset{width=10cm}
        \begin{tikzpicture}[baseline=(current bounding box.center)]
            \begin{axis}[
                %title={\textbf{Stress-Strain Curve}},
                xlabel={Strain},
                ylabel={Stress (ksi)},
                xmin=0, xmax=0.015,
                ymin=0, ymax=150,
                xtick={0,0.005,0.01,0.015},
                ytick={0,50,100,150},
                ymajorgrids=true,
                grid style=dashed,
                legend pos = north west,
                legend cell align={left},
                xticklabel style={
                    /pgf/number format/fixed,
                    /pgf/number format/precision=5
                },
                scaled x ticks=false
            ]
            
            \addplot[smooth, red] table [x=strain, y=stress] {1.csv};
            \addplot[blue, dotted, domain=0:0.015] {29701.46*x-59.4029};
            %\addplot[blue] table [x=strain, y=stress_th] {1.csv};
            \addlegendentry{Stress-Strain Curve}
            \addlegendentry{0.2\% Offset}
            \end{axis}
        \end{tikzpicture}
    \end{center}
    The intersection point is the point at which the strain is approximately 0.0057. This value gets plugged into the equation for the 0.2\% offset line to determine the yield stress.
    \[y=29701.46\times x-59.4029=29701.46\times0.0057-59.4029=\boxed{109.89\text{ ksi}}\]
    \newpage
    \section*{Chapter 2}
    \subsection*{Question 2-4} 
    \noindent The question gives a dead load of 18 kips and a live load of 2 kips. For LRFD, the controlling AISC load combinations are as follows:
    \begin{enumerate}
        \item \(1.4D\) 
        \item \(1.2D+1.6L+0.5\,(L_r\text{ or }S\text{ or }R)\)
        \item \(1.2D+1.6\,(L_r\text{ or }S\text{ or }R)+(0.5L\text{ or }0.5W)\)
        \item \(1.2D+1.0W+0.5\,(L_r\text{ or }S\text{ or }R)\)
        \item \(0.9D+1.0W\)
    \end{enumerate}
    Combinations 2, 3, 4, and 5 cannot be used since only the dead load and live load are given. The controlling load combination is therefore \(\boxed{\text{Combination 1}}\). As a result, the maximum factored load for LRFD is found as follows:
    \[1.4\times 18\text{ kips}=\boxed{25.2\text{ kips}}\] 
    \noindent For ASD, the controlling AISC load combinations are as follows: 
    \begin{enumerate}
        \item \(D\) 
        \item \(D+L\)
        \item \(D+(L_r\text{ or }S\text{ or }R)\)
        \item \(D+0.75L+0.75\,(L_r\text{ or }S\text{ or }R)\)
        \item \(D+0.6W\)
        \item \(D+0.75L+0.75\,(0.6W)+0.75\,(L_r\text{ or }S\text{ or }R)\) 
        \item \(0.6D+0.6W\)
    \end{enumerate}
    \noindent Combinations 3, 4, 5, 6, and 7 cannot be used since only the dead load and live load are given. The controlling AISC load combination is therefore \(\boxed{\text{Combination 2}}\), as it is the greater of the only two possible load combinations that can be used. As a result, the maximum factored load for ASD is found as follows: 
    \[18\text{ kips}+2\text{ kips}=\boxed{20\text{ kips}}\]
    \newpage 
    \subsection*{Question 2-5}
    \noindent The question gives a dead load of 21 psf, a roof live load of 12 psf, a snow load of 13.5 psf, and a wind load of 22 psf. The wind load is taken as 0 psf since it is acting upwards. For LRFD, the controlling AISC load combinations are as follows:
    \begin{enumerate}
        \item \(1.4D = 1.4\times 21\text{ psf}=29.4\text{ psf}\) 
        \item \(1.2D+1.6L+0.5\,(L_r\text{ or }S\text{ or }R)\) (no live load given)
        \item \(1.2D+1.6\,(L_r\text{ or }S\text{ or }R)+(0.5L\text{ or }0.5W)=1.2\times 21\text{ psf}+1.6\times13.5\text{ psf}+0.5\times 0\text{ psf}=46.8\text{ psf} \)
        \item \(1.2D+1.0W+0.5\,(L_r\text{ or }S\text{ or }R)=1.2\times 21\text{ psf}+0\text{ psf}+0.5\times13.5\text{ psf}=31.95\text{ psf}\)
        \item \(0.9D+1.0W=0.9\times 21\text{ psf}+0\text{ psf}=18.9\text{ psf}\)
    \end{enumerate}
    It is found that the controlling AISC load combination for LRFD is \(\boxed{\text{Combination 3}}\), with the greatest maximum factored load of \(\boxed{46.8\text{ psf}}\). 
    \\\\For ASD, the controlling AISC load combinations are as follows: 
    \begin{enumerate}
        \item \(D=21\text{ psf}\) 
        \item \(D+L\) (no live load given)
        \item \(D+(L_r\text{ or }S\text{ or }R)=21\text{ psf}+13.5\text{ psf}=34.5\text{ psf}\)
        \item \(D+0.75L+0.75\,(L_r\text{ or }S\text{ or }R)\) (no live load given)
        \item \(D+0.6W=21\text{ psf}+0.6\times0\text{ psf}=21\text{ psf}\)
        \item \(D+0.75L+0.75\,(0.6W)+0.75\,(L_r\text{ or }S\text{ or }R)\) (no live load given)
        \item \(0.6D+0.6W=0.6\times21\text{ psf}+0.6\times0\text{ psf}=12.6\text{ psf}\)
    \end{enumerate}
    It is found that the controlling AISC load combination for ASD is \(\boxed{\text{Combination 3}}\), with the greatest maximum factored load of \(\boxed{34.5\text{ psf}}\).
\end{document}