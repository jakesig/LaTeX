\documentclass{article}

\usepackage[english]{babel}

\usepackage[letterpaper,top=2.5cm,bottom=2.5cm,left=2.5cm,right=2.5cm,marginparwidth=1.75cm]{geometry}

\usepackage{amsmath}
\usepackage{graphicx}
\usepackage[colorlinks=true, allcolors=blue]{hyperref}
\usepackage{cancel, fancyhdr, textcomp, subfig}
\usepackage{bm, pgfplots}
\usepackage[export]{adjustbox}

\renewcommand{\thesubfigure}{}
\captionsetup[subfigure]{labelformat=simple, labelsep=colon}
\pagestyle{fancy}
\fancyhf{}
\rhead{Jacob Sigman\\1/30/23}
\lhead{CE-331\\Homework}
\cfoot{\thepage}
\renewcommand\arraystretch{1.5}
\renewcommand{\headrulewidth}{1.5pt}
\setlength{\headheight}{22.6pt}
\begin{document}
\subsection*{Question 2.4}
The following table can be obtained to determine the percent finer than each sieve.
\begin{center}
    \begin{tabular}{|ccccc|} 
        \hline
        \textbf{US sieve} & \textbf{Opening (mm)} & \textbf{Retained Mass (g)} & \textbf{Cumulative Mass (g)} & \textbf{Percent Finer}  \\ 
        \hline
        4                 & 4.75                  & 0                          & 0                            & 100.00\%                \\
        6                 & 3.35                  & 30                         & 30                           & 94.00\%                 \\
        10                & 2                     & 48.7                       & 78.7                         & 84.26\%                 \\
        20                & 0.85                  & 127.3                      & 206                          & 58.80\%                 \\
        40                & 0.425                 & 96.8                       & 302.8                        & 39.44\%                 \\
        60                & 0.25                  & 76.6                       & 379.4                        & 24.12\%                 \\
        100               & 0.15                  & 55.2                       & 434.6                        & 13.08\%                 \\
        200               & 0.075                 & 43.4                       & 478                          & 4.40\%                  \\
        Pan               & -                     & 22                         & 500                          & 0.00\%                  \\
        \hline
    \end{tabular}
\end{center}
Below is the grain-size distribution curve.
\begin{center}
    \pgfplotsset{width=10cm}
    \begin{tikzpicture}[baseline=(current bounding box.center)]
        \begin{axis}[
            %title={\textbf{Stress-Strain Curve}},
            xlabel={Opening (mm)},
            ylabel={Percent Finer},
            xmin=0, xmax=5,
            ymin=0, ymax=100,
            xtick={0,1,2,3,4,5},
            ytick={0,25,50,75,100},
            ymajorgrids=true,
            grid style=dashed,
            legend pos = north west,
            legend cell align={left},
            x dir=reverse
        ]
        
        \addplot[smooth, red] table [x=opening, y=finer] {1.csv};
        \end{axis}
    \end{tikzpicture}
\end{center}
\(D_{10}\), \(D_{30}\), and \(D_{60}\) can be calculated using linear interpolation (done in Excel).
\[D_{10}=\boxed{0.12\text{ mm}}\]
\[D_{30}=\boxed{0.32\text{ mm}}\] 
\[D_{60}=\boxed{0.9\text{ mm}}\] 
The uniformity coefficient is calculated as follows.
\[C_u=\frac{D_{60}}{D_{10}}=\frac{0.9\text{ mm}}{0.12\text{ mm}}=\boxed{7.33}\]
The coefficient of gradation is calculated as follows.
\[C_c=\frac{D_{30}^2}{D_{60}\times D_{10}}=\frac{(0.32\text{ mm})^2}{(0.9\text{ mm})\times (0.12\text{ mm})}=\boxed{0.9}\]
\subsection*{Question 2.8}
Below is the grain-size distribution curve plotted on a logarithmic scale.
\begin{center}
    \pgfplotsset{width=10cm}
    \begin{tikzpicture}[baseline=(current bounding box.center)]
        \begin{axis}[
            %title={\textbf{Stress-Strain Curve}},
            xlabel={Size (mm)},
            ylabel={Percent Finer},
            xmode=log,
            xmin=0, xmax=1,
            ymin=0, ymax=100,
            xtick={0.001,0.01,0.1,1},
            ytick={0,25,50,75,100},
            ymajorgrids=true,
            grid style=dashed,
            legend pos = north west,
            legend cell align={left},
            x dir=reverse
        ]
        
        \addplot[smooth, red] table [x=size, y=finer] {2.csv};
        \end{axis}
    \end{tikzpicture}
\end{center}
First, the values according to USDA system are used, these values were calculated using linear interpolation (done in Excel).
\begin{center}
    Passing 2 mm: 100\% \hspace{3mm} Passing 0.05 mm: 90.4\% \hspace{3mm} Passing 0.002 mm: 41.2\%
\end{center}
The percentages of gravel, sand, slit, and clay are calculated as follows:
\[\%_\text{Gravel}=100\%-100\%=\boxed{0\%}\] 
\[\%_\text{Sand}=100\%-90.4\%=\boxed{9.6\%}\] 
\[\%_\text{Slit}=90.4\%-41.2\%=\boxed{49.3\%}\] 
\[\%_\text{Clay}=\boxed{41.2\%}\] 
Next, the values according to AASHTO system are used, these values were calculated using linear interpolation (done in Excel).
\begin{center}
    Passing 2 mm: 100\% \hspace{3mm} Passing 0.075 mm: 91.1\% \hspace{3mm} Passing 0.002 mm: 41.2\%
\end{center}
The percentages of gravel, sand, slit, and clay are calculated as follows:
\[\%_\text{Gravel}=100\%-100\%=\boxed{0\%}\] 
\[\%_\text{Sand}=100\%-91.1\%=\boxed{8.9\%}\] 
\[\%_\text{Slit}=91.1\%-41.2\%=\boxed{49.9\%}\] 
\[\%_\text{Clay}=\boxed{41.2\%}\] 
\newpage
\subsection*{Question 2.10}
The \(K\) was determined to be 0.1321 from Table 2.9 for the given specific gravity of 2.6 and the given temperature of 24$^{\circ}$C. Equation 2.9 is used to determine $L$.
\[L=16.29-0.164\times R=16.29-0.164\times 43=9.238\text{ cm}\] 
\(D\) is determined using the following equation, using the determined $L$ and $K$ values along with a $t$ value of 60 minutes.
\[D=K\sqrt{\frac{L}{t}}=0.1321\sqrt{\frac{9.238\text{ cm}}{60\text{ min}}}=\boxed{0.052\text{ mm}}\]
\end{document}