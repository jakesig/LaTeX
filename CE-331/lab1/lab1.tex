\documentclass{article}

\usepackage[english]{babel}
\usepackage[letterpaper,top=2.5cm,bottom=2.5cm,left=2.5cm,right=2.5cm,marginparwidth=1.75cm]{geometry}
\usepackage{amsmath, graphicx, tikz, pgfplots, multirow, newlfont, gensymb, indentfirst, bm, setspace, xurl}
\usepackage[usestackEOL]{stackengine}
\usepackage[export]{adjustbox}
\usepackage{fancyhdr}
\pagestyle{fancy}
\fancyhf{}
\rhead{Part 1\\4/14/23}
\lhead{CE-331\\Group 2}
\cfoot{\thepage}
\renewcommand{\headrulewidth}{1.5pt}
\setlength{\headheight}{22.6pt}
\usepackage[colorlinks=true, allcolors=black]{hyperref}
\setlength\parindent{24pt}

\begin{document}
\begin{titlepage}
    \begin{center}
    {{\Large{\textsc{The Cooper Union for the Advancement of Science and Art}}}} \rule[0.1cm]{15.8cm}{0.1mm}
    \rule[0.5cm]{15.8cm}{0.6mm}
    {\small{\bf DEPARTMENT OF CIVIL AND ENVIRONMENTAL ENGINEERING}}\\
    {\footnotesize{STRUCTURAL ENGINEERING LABORATORY}}
    \end{center}
    \vspace{15mm}
    \begin{center}
    {\large{\bf LAB 5\\}}
    \vspace{5mm}
    {\Large{\bf COMPRESSIVE TESTING OF\\}}
    \vspace{2mm}
    {\Large{\bf STANDARD CONCRETE CYLINDER}}
    \end{center}
    \vspace{35mm}
    \par
    \noindent
    \hfill
    \vspace{20mm}
    \begin{center}
    {\large{ {\bf Group 2} \\ { Jenna Manfredi\hspace{5mm}David Madrigal\hspace{5mm}Gila Rosenzweig\\Nicole Shamayev\hspace{5mm}Jake Sigman}}}
    \vspace{40mm}
    {\large {\bf \\CE-321 \\ 12/14/22 \\}}
    \vspace{15mm}
    {\normalsize{Professor Tzavelis \\ Avery Kugler \\ Lionel Gilliar-Schoenenberger \\ Crystal Woo}}
    \end{center}
\end{titlepage}
\newpage
\doublespacing
\tableofcontents
\newpage
\addcontentsline{toc}{section}{List of Tables}
\listoftables
\addcontentsline{toc}{section}{List of Figures}
\listoffigures
\newpage
\section{Overview} 
\par This report details the properties of a soil sample obtained from Manalapan, New Jersey. The experiments run and the results are to be used for consideration in the design of a flexible pavement project to be executed in the area. A summary of results is presented in the following table. 
\begin{center}
\begin{tabular}{|l|r|}
    \hline 
    $\bm{\textbf{Specific Gravity at } 20^\circ\textbf{C}}$ & 2.714\\\hline
    \textbf{AASHTO Classification} & A-1-b (0)\\\hline 
    \textbf{USCS Classification} & GP-GM\\\hline 
    \textbf{FAA Classification} & E-5 (F1)\\\hline 
    \textbf{Coefficient of Gradation, $\bm{C_c}$} & 0.075\\\hline 
    \textbf{Liquid Limit} & 26.84\%\\\hline 
    \textbf{Plastic Limit} & 34.6\%\\\hline 
    \textbf{Dry Unit Weight} & 0\\\hline 
    \textbf{Optimum Moisture Content} & 8.55\%\\\hline 
    \textbf{Dry Unit Weight @ 75\% Relative Density} &0 \\\hline 
    \textbf{Moisture Content @ 75\% Relative Density} & 0\\\hline 
    \textbf{CBR} & 0\\\hline 
    \textbf{Soaked CBR} & 0\\\hline
\end{tabular}
\vspace{3mm}
\addcontentsline{lot}{table}{Table 1: Overview}
\emph{\\Table 1: Overview}
\end{center}
\newpage
\section{Experimental Results}
\subsection{Specific Gravity Determination}
\subsubsection{Overview}
\par Determining the specific gravity of soil is important in geotechnical engineering. The specific gravity of any substance is defined as the ratio of its density to the density of water. In this experiment, the specific gravity is determined by comparing the weight of a given volume of soil to the weight of the same volume of water. This provides valuable information about the composition of the soil its sustainability for various engineering applications. To perform the experiment, a soil sample is collected and oven-dried to remove any excess moisture. Then, a known volume of the sample is weighed. The same volume of water is also weighed. The specific gravity is calculated by dividing the weight of the dry soil by the weight of the water. 
\subsubsection{Results}
\begin{center}
    \begin{tabular}{|l|r|}
        \hline 
        $\bm{G_s \textbf{ at } 20^\circ\textbf{C}}$ & 2.714\\\hline
        $\bm{G_s \textbf{ at } 4^\circ\textbf{C}}$ & 2.709\\\hline
    \end{tabular}
    \vspace{3mm}
    \addcontentsline{lot}{table}{Table 2: Specific Gravities of Soil Sample at Various Temperatures}
    \emph{\\Table 2: Specific Gravities of Soil Sample at Various Temperatures}
\end{center}
\par As seen in the table above, the specific gravity of the soil sample at 20 $^\circ$C is 2.72. This means that at 20 $^\circ$C, the density of the soil is 2.72 times the density of water. Additionally, the specific gravity of the soil sample at 4 $^\circ$C is 2.71, meaning that at 4 $^\circ$C, the density of the soil is 2.71 times the density of water. While this is a minor difference from 20 $^\circ$C, it is still important to the experiment as the densities of both soil and water can vary at different temperatures.  
\newpage
\subsection{Grain Size Distribution}
\subsubsection{Overview}
\par In determining grain size distribution, the soil sample was run through a series of sieves to separate the fine and coarse fractions (gravel and sand), which for this series of experiments is defined to be passing No. 10 sieve and retained on No. 10 sieve, respectively. The sieves used for this analysis were: 1.500 in, 1.000 in, 0.750 in, 0.500 in, 0.375 in, No. 4, No. 10, No. 16, No. 30, No. 60, No. 100, and No. 200. 
\par First, the entire quantity was separated by both No. 4 and No. 10 sieves into +No. 4, +No. 10, and -No. 10 fractions. The coarse fraction is all +No. 10 material (+No. 4 and -No. 4 +No. 10) and is worked through a nest of sieves comprised of: 1.5”, 1.0”, 3/4”, 1/2”, 3/8”, No. 4, and No. 10 sieves. First, the +No. 4 material is worked through the sieves. Any material that passed through the No. 4 sieve was added to the portion of the sample that is -No. 4 +No. 10; any material that is found to be -No. 10 should be noted as a correction.  
\par Once oven dried, the -No. 10 material is worked through a nest of sieves comprised of No. 16, No. 30, No. 60, No. 100, and No. 200. See Appendix for recorded weights of soils retained on each sieve.  
\par A hydrometer analysis was performed on 75.02 grams (wet soil) of the fine fraction to which distilled water was added such that the soil was thoroughly wetted. The wetted soil was then mixed with 100 ml of dispersing agent, loosely covered and allowed to soak for at least 18 hours. This mixture was then transferred to a 1000 ml graduated cylinder, using a spatula and wash bottle as needed. A compressed air mechanical mixer was used to give further dispersion to the sample and mixed at a pressure of 25 psi. for an amount of time corresponding to the plastic index. After the mixing period, pressure was reduced to 1 psi. The mixer tube was unseated and lifted from the graduated cylinder at the same time as it was washed with distilled water to return any soil grains to the graduated cylinder. The graduate was then mixed to create a soil suspension of uniform density; a stirring rod was pumped vigorously for 40 seconds up and down through the entire depth of the suspension, then slowly pumped in the upper half of the suspension for 15 seconds, and finally moved slowly up through the suspension and withdrawn from the graduate for the last 5 seconds. The hydrometer was immediately inserted to a depth slightly below its floating depth and allowed to freely float to rest. Readings were taken at elapsed times of 1/2, 1, and 2 minutes, with the hydrometer remaining continuously in suspension for those two minutes. The hydrometer was then removed, dried, and reinserted to attain three reading at each time; the average of the 1/2, 1, and 2 minute readings was used as the data. Additional hydrometer readings at 4, 8, 15, and 30 minutes, as well as at 1, 2, 4, and 24 hours of elapsed time were recorded, as well as temperature at 2 minutes and all subsequent readings.  
\par Upon the last hydrometer reading, the soil suspension was transferred to a No. 200 sieve, and washed with tap water until the water ran clear. Material retained on the sieve was dried in the oven. The hydrometer and 1000 ml graduated cylinder were calibrated as follows: cross sectional area of the graduate was found and recorded; the volume of the hydrometer bulb was found and recorded; the distance from the neck of the bulb to a calibration mark on the hydrometer stem was measured and recorded along with the corresponding hydrometer reading; the distance from the neck of the bulb to the tip is measured and recorded; the value of the meniscus correction was found and recorded; and the value of the density correction was found and recorded. Calibrated hydrometer readings can be found in the Appendix.  
\newpage  
\subsubsection{Results}
\par When hydrometer readings are included in the grain size distribution analysis, as calculated in the provided spreadsheet, the table below describes the soil properties. If, however, a grain size distribution curve is produced according to mass retained on each sieve, those properties are also shown below. 
\begin{center}  
    \begin{tabular}{|l|cc|} 
        \hline
        \textbf{Property} & \textbf{Standard} & \textbf{Using Retained Mass} \\
        \hline
        \textbf{Effective Size, $\bm{D_{10}}$ (mm)}           & 0.51  & 4.75     \\ 
        \hline
        \textbf{$\bm{D_{30}}$ (mm)}           & 0.2 & 15      \\ 
        \hline
        \textbf{Median Size, $\bm{D_{50}}$ (mm)}              & 5 & 18      \\ 
        \hline
        \textbf{$\bm{D_{60}}$ (mm)}           & 10.5 & 4.21      \\ 
        \hline
        \textbf{Uniformity Coefficient}   & 205.88 & 4.21      \\ 
        \hline
        \textbf{Coefficient of Gradation} & 0.075 & 2.37      \\ 
        \hline
        \textbf{\% Coarse Gravel}         & 0 & 0      \\ 
        \hline
        \textbf{\% Medium Gravel}         & 45.055 & 80.66      \\ 
        \hline
        \textbf{\% Fine Gravel}           & 5.209 & 9.32      \\ 
        \hline
        \textbf{\% Coarse Sand}           & 0.133 & 0.738      \\ 
        \hline
        \textbf{\% Medium Sand}           & 0.4922 & 0.121      \\ 
        \hline
        \textbf{\% Fine Sand}             & 37.37 & 9.155     \\ 
        \hline
        \textbf{\% Silt and Clay}         & 11.74 & 0      \\
        \hline
    \end{tabular}
    \vspace{3mm}
    \addcontentsline{lot}{table}{Table 3: Grain Size Distribution Properties}
    \emph{\\Table 3: Grain Size Distribution Properties\\}
\end{center}
\par In this table, coarse gravel is defined as soil of diameter greater than 38 mm, medium gravel is greater than 12.5mm, fine gravel greater than 4.75 mm. Coarse sand is greater than 1.18mm, medium sand greater than 0.6 mm, and fine sand greater than 0.075 mm. Silt and clay are defined as smaller than 0.075 mm.

\begin{center}
    \pgfplotsset{width=10cm}
    \begin{tikzpicture}[baseline=(current bounding box.center)]
        \begin{axis}[
            xlabel={Grain Size (mm)},
            ylabel={Percent Finer (\%)},
            xmin=0, xmax=40,
            ymin=0, ymax=100,
            xmode = log,
            x dir = reverse,
            xtick={10, 1, 0.1, 0.01, 0.001},
            ytick={25, 50, 75, 100},
            ymajorgrids=true,
            grid style=dashed,
            legend pos = north east,
            legend cell align={left},
        ]
        \addplot[smooth, red] table [x=x, y=y] {2.csv};
        \end{axis}
    \end{tikzpicture}
    \vspace{3mm}
    \addcontentsline{lof}{figure}{Figure 1: Grain Size Distribution - Logarithmic Scale}
    \emph{\\Figure 1: Grain Size Distribution - Logarithmic Scale\\}
    \vspace{5mm}
    \begin{tikzpicture}[baseline=(current bounding box.center)]
        \begin{axis}[
            xlabel={Grain Size (mm)},
            ylabel={Percent Finer (\%)},
            xmin=0, xmax=40,
            ymin=0, ymax=100,
            x dir = reverse,
            xtick={10, 0.001},
            ytick={25, 50, 75, 100},
            ymajorgrids=true,
            grid style=dashed,
            legend pos = north east,
            legend cell align={left},
        ]
        \addplot[smooth, red] table [x=x, y=y] {2.csv};
        \end{axis}
    \end{tikzpicture}
    \vspace{3mm}
    \addcontentsline{lof}{figure}{Figure 2: Grain Size Distribution - Linear Scale}
    \emph{\\Figure 2: Grain Size Distribution - Linear Scale}
    \begin{tikzpicture}[baseline=(current bounding box.center)]
        \begin{axis}[
            xlabel={Grain Size (mm)},
            ylabel={Percent Finer (\%)},
            xmin=0, xmax=100,
            ymin=0, ymax=100,
            x dir = reverse,
            xmode = log,
            xtick={100, 10, 1, 0.1, 0.01},
            ytick={25, 50, 75, 100},
            ymajorgrids=true,
            grid style=dashed,
            legend pos = north east,
            legend cell align={left},
        ]
        \addplot[smooth, red] table [x=d, y=p] {7.csv};
        \end{axis}
    \end{tikzpicture}
    \vspace{3mm}
    \addcontentsline{lof}{figure}{Figure 3: Grain Size Distribution - Sieve Analysis}
    \emph{\\Figure 3: Grain Size Distribution - Sieve Analysis\\}
    \vspace{5mm}
    \begin{tikzpicture}[baseline=(current bounding box.center)]
        \begin{axis}[
            xlabel={Distance (cm)},
            ylabel={$R_H+c_m$},
            xmin=13.8, xmax=15.2,
            ymin=10, ymax=10.14,
            xtick={13.8, 14.2, 14.6, 15.2},
            ytick={10.02, 10.04, 10.06, 10.08, 10.1, 10.12, 10.14},
            yticklabels={1.002, 1.004, 1.006, 1.008, 1.01, 1.012, 1.014},
            ymajorgrids=true,
            grid style=dashed,
            legend pos = north east,
            legend cell align={left},
        ]
        \addplot[only marks, red] table [x=y1, y=x1_mod] {8.csv};
        \addplot[only marks, blue] table [x=y2, y=x2_mod] {9.csv};
        \addlegendentry{First Two Minutes};
        \addlegendentry{Subsequent Readings};
        \addplot[dashed, red, domain=13.8:15.2] {10*(-0.0039*x+1.0676)};
        \addplot[dashed, blue, domain=13.8:15.2] {10*(-0.0045*x+1.0712)};
        \end{axis}
    \end{tikzpicture}
    \vspace{3mm}
    \addcontentsline{lof}{figure}{Figure 4: Hydrometer Calibration Graph}
    \emph{\\Figure 4: Hydrometer Calibration Graph}
\end{center}
\newpage
\subsection{Atterberg Limits}
\subsubsection{Overview}
\par This experiment was separated into two parts, both asynchronous from one another. The first part of this experiment was to identify the liquid limit of the soil by placing a sample within a grooving machine called a Casagrande grooving tool. First, place a sample within the bowl, making sure that the sample is compacted within the bowl. Once compacted, `blow', or tap, the sample repeatedly at a constant pace until a clear grove has been created, around 1/2 inch in depth. Record the number of blows that it took to create the marking. After each trial, add either water or soil, depending on the moisture content of the current sample. If it is dry, add water. If it is wet, add more soil. Repeat this for a total of five times, making sure that the number of blows increase per trial and a sample has been collected to be weighted and dried in the oven. Dry all samples in a weighted evaporating dish and cover with a weighted watch glass. 
\par The second part is focused on finding the plastic limit. Taking a sample of the given soil, continuously mix in water until an oval-shaped ball has been formed and the shape can be sustained. Once the soil has been kneaded to the preferred consistency, take a portion of the soil and begin rolling the soil out into a thin snake-like shape. If it breaks before reaching 1/8 inch in diameter, record the weight of the soil and place it in an oven to dehydrate. If it can sustain a snake-like shape with a diameter smaller than 1/8 inch, repeat the rolling process until it breaks before the size limit. Rerolling will draw out moisture, allowing the soil to reach its desired plasticity. Repeat this process two more times, for a total of three trials. As before, place all samples into a weighted evaporating dish, cover with a weighted watch glass, and heat in the oven.  
\par Once all the samples have been dried out, place all samples into the desiccator to cool down without introducing new moisture into the samples. Once the samples are cool to the touch, record the weight of the samples.  
\newpage
\subsubsection{Results}
\begin{center}
    \begin{tabular}{|l|r|}
        \hline 
        \textbf{Liquid Limit} & 26.84\%\\\hline
        \textbf{Plastic Limit} & 34.6\%\\\hline
        \textbf{Flow Index} & 17.65\\\hline
        \textbf{Liquid Index} & 1\\\hline
    \end{tabular}
    \vspace{3mm}
    \addcontentsline{lot}{table}{Table 4: Atterberg Limits}
    \emph{\\Table 4: Atterberg Limits}
\end{center}
\begin{center}  
    \pgfplotsset{width=10cm}
    \begin{tikzpicture}[baseline=(current bounding box.center)]
        \begin{axis}[
            xlabel={\# of Blows},
            ylabel={Moisture Content (\%)},
            xmin=0, xmax=100,
            ymin=24, ymax=30,
            xmode=log,
            xtick={10, 100},
            ytick={24, 26, 28, 30},
            ymajorgrids=true,
            grid style=dashed,
            legend pos = north east,
            legend cell align={left},
        ]
        
        \addplot +[blue, mark=none] coordinates {(25, 24) (25, 30)};
        \addplot[only marks, red] table [x=nb, y=mc] {1.csv};
        \addplot[dashed, red, domain=20:49] {33.322*(2.718281828)^(-0.006*x)};
        \addlegendentry{$x=25$ blows}
        
        \end{axis}
    \end{tikzpicture}
    \vspace{3mm}
    \addcontentsline{lof}{figure}{Figure 5: Flow Index}
    \emph{\\Figure 5: Flow Index}
\end{center}
\par In the first experiment, results and data calculated the Liquid Limit to be 26.84, meaning that, starting with the sample, if the soil becomes 27\% liquid, the sample will behave as a liquid more than a solid. This result was achieved during laboratory experiments, by using a Casagrande grooving tool, as well as precise measurements of the soil sample's weights, both wet and dry, as well as giving proper attention to the varying weights of the containers used.  
\par Following the first experiment, the second experiment identified the Plastic Limit as 34.6. By using the Liquid Limit and subtracting it with the Plastic Limit, we obtained the Plastic Index, and this calculation resulted in a negative value. For Plastic Index, since it cannot be a negative value, when the resulted value is less than zero, Plastic Index equals zero. This makes this sample's degree of plasticity non-plastic. Furthermore, the Activity value of this sample is also zero, due to both the Plastic Index and amount of clay to be both zero.  
\par For the Flow Index, we obtain a value of 17.65. Note, however, that the first data point was omitted from the calculations and graph, due to its obscure results, which deems it as an outlier. Referring to Figure (insert figure of Blows per water contents), we see that the first data is significantly lower than the graph's natural trendline. This is speculated to be human error, as familiarity and experience builds per trial, the first attempt would most likely hold the highest chances of mistakes.  
\par For the Liquid Index, the results indicate a value of 1. This value dictates the ratio between plasticity and liquid. A value of 1 would indicate an even ratio, resulting in a non-plastic sample, due to any plasticity property not being able to activate and be of use.
\newpage
\subsection{Compaction Test}

\subsubsection{Overview}
\par Soil compaction is the process of densifying soil by reducing the amount of air present in the soil. The soil compaction test completed in this experiment was the Standard Proctor Test. In the Standard Proctor Test, soil was compacted in a standard cylindrical mold having a volume of 1/30 cubic feet and a diameter of 4 inches. This mold was first weighed empty with an electronic balance to get its initial weight before filling it with two to three inches of soil taken from the -4 material. Once the mold was filled, it was placed in the compaction box to be uniformly hammered approximately 25 times with the compacting hammer. This hammer was also weighed prior to compacting and the height of the freefall blows was recorded in order to calculate the energy used in compaction.  
\par After completing the first layer, another two layers are added by the same process of uniform hammer compacting, ensuring equal layer thickness and that the final compacted soil extends slightly over the rim of the mold. With the mold filled, the collar was carefully removed and the mold plus soil weight was recorded with the same electronic balance used previously. First centering a 1.375" diameter loading block on the soil sample surface, the soil sample was loaded into a compressing machine with a dial indicator to measure soil penetration as the sample is compressed. Compression load was applied at a strain rate of 0.06 in/min and the load at the time of 0.025, 0.05, 0.075, 0.1, 0.2, 0.3, 0.4 and 0.5 inches of penetration was recorded.  
\par Once the compression loading process was complete, the soil was separated from the mold using a hydraulic sample extruder and reduced using a spatula until the center was exposed. From the center, a small sample was removed in order to be tested for moisture content and placed in a previously weighed crystallizing dish and watch glass. Finally, the dish and soil are weighed together before being placed in an oven to be dried. This process was repeated for 2 more trials using soil of increasing moisture content. In between trials, the moisture content of the soil is increased by uniformly spraying and mixing water into the tray containing the initially measured 3000 grams of -4 soil material. Once the soil samples taken from the cored compaction material were dried, their final weights were recorded in order to find moisture content.  
\subsubsection{Results}
\begin{center}  
    \begin{tabular}{|l|r|} 
        \hline
        \textbf{Optimum Moisture Content (\%)}           & 8.55       \\ 
        \hline
        \textbf{Maximum Density (lb/ft$\bm{^3}$)}               & 104.85       \\ 
        \hline
        \textbf{Minimum Density (lb/ft$\bm{^3}$)}   & 0       \\ 
        \hline
        \textbf{Relative Density at 75\%}  & 0       \\ 
        \hline
        \textbf{Moisture Content at 75\%}          & 0       \\ 
        \hline
        \textbf{CBR Design Value}          & 20.94       \\ 
        \hline
    \end{tabular}
    \vspace{3mm}
    \addcontentsline{lot}{table}{Table 5: Compaction Test Results}
    \emph{\\Table 5: Compaction Test Results\\}
    \vspace{5mm}
    \pgfplotsset{width=13cm}
    \begin{tikzpicture}[baseline=(current bounding box.center)]
        \begin{axis}[
            xlabel={Moisture Content (\%)},
            ylabel={Dry Unit Weight (lb/ft$^3$)},
            xmin=0, xmax=50,
            ymin=0, ymax=200,
            xtick={10,20,30,40,50},
            ytick={50, 100, 150,200},
            ymajorgrids=true,
            grid style=dashed,
            legend pos = north east,
            legend cell align={left},
        ]
        
        \addplot[only marks, red] table [x=x, y=porosity] {3.csv};
        \addplot[only marks, blue] table [x=x, y=dry] {3.csv};
        \addplot[only marks, green] table [x=x, y=wet] {3.csv};
        \addplot[only marks, orange] table [x=x, y=100] {3.csv};
        \addplot[only marks, purple] table [x=x, y=80] {3.csv};
        \addplot +[only marks, black, mark options={fill=black}] coordinates {(8.55, 104.85)};
        \addlegendentry{Porosity};
        \addlegendentry{Dry Unit Weight};
        \addlegendentry{Wet Unit Weight};
        \addlegendentry{Theoretical Dry Unit Weight at $S=100$};
        \addlegendentry{Theoretical Dry Unit Weight at $S=80$};
        \addlegendentry{Optimum Moisture Content and Maximum Density}
        \addplot[red, domain=8:39] {0.0158*x^2-0.2708*x+39.239};
        \addplot[purple, domain=8:39] {0.0372*x^2-3.6758*x+159.83};
        \addplot[blue, domain=8:39] {-0.0268*x^2+0.4585*x+102.89};
        \addplot[orange, domain=8:39] {0.0307*x^2-3.2656*x+163.01};
        \addplot[green, domain=8:39] {-0.0392*x^2+1.7855*x+101.53};
        \addplot +[black, mark=none, dashed] coordinates {(8.55, 104.85) (8.55, 0)};
        \addplot +[black, mark=none, dashed] coordinates {(8.55, 104.85) (0, 104.85)};
       
        \end{axis}
    \end{tikzpicture}
    \vspace{3mm}
    \addcontentsline{lof}{figure}{Figure 6: Standard Proctor Test Results}
    \emph{\\Figure 6: Standard Proctor Test Results\\}
    \vspace{5mm}
    \pgfplotsset{width=10cm}
    \begin{tikzpicture}[baseline=(current bounding box.center)]
        \begin{axis}[
            xlabel={Penetration (in)},
            ylabel={Stress (psi)},
            xmin=0, xmax=0.3,
            ymin=0, ymax=500,
            xtick={0.1,0.2,0.3},
            ytick={0, 100, 200, 300, 400, 500},
            ymajorgrids=true,
            grid style=dashed,
            legend pos = north east,
            legend cell align={left},
        ]
        
        \addplot[only marks, red] table [x=x, y=y] {4.csv};
        \addplot[dashed, red, domain=0:0.3] {-10363*x^2+4260.4*x};
        \end{axis}
    \end{tikzpicture}
    \vspace{3mm}
    \addcontentsline{lof}{figure}{Figure 7: First Determination}
    \emph{\\Figure 7: First Determination\\}
    \vspace{5mm}
    \pgfplotsset{width=10cm}
    \begin{tikzpicture}[baseline=(current bounding box.center)]
        \begin{axis}[
            xlabel={Penetration (in)},
            ylabel={Stress (psi)},
            xmin=0, xmax=0.5,
            ymin=0, ymax=500,
            xtick={0.1,0.2,0.3,0.4,0.5},
            ytick={0, 100, 200, 300, 400, 500},
            ymajorgrids=true,
            grid style=dashed,
            legend pos = north east,
            legend cell align={left},
        ]
        
        \addplot[only marks, red] table [x=x, y=y] {5.csv};
        \addplot[dashed, red, domain=0:0.5] {-1121.1*x^2+1489.9*x};
        \end{axis}
    \end{tikzpicture}
    \vspace{3mm}
    \addcontentsline{lof}{figure}{Figure 8: Second Determination}
    \emph{\\Figure 8: Second Determination\\}
    \vspace{5mm}
    \pgfplotsset{width=10cm}
    \begin{tikzpicture}[baseline=(current bounding box.center)]
        \begin{axis}[
            xlabel={Penetration (in)},
            ylabel={Stress (psi)},
            xmin=0, xmax=0.5,
            ymin=0, ymax=18,
            xtick={0.1,0.2,0.3,0.4,0.5},
            ytick={0, 6, 12, 18},
            ymajorgrids=true,
            grid style=dashed,
            legend pos = north east,
            legend cell align={left},
        ]
        
        \addplot[only marks, red] table [x=x, y=y] {6.csv};
        \addplot[dashed, red, domain=0:0.5] {-105.61*x^3+74.587*x^2+21.964*x};
        \addplot +[only marks, blue, mark options={fill=blue}] coordinates {(0.1, 2.771916147)};
        \addplot[dashed, blue, domain=0:0.1] {40.571*x-1.2171};
        \end{axis}
    \end{tikzpicture}
    \vspace{3mm}
    \addcontentsline{lof}{figure}{Figure 9: Third Determination}
    \emph{\\Figure 9: Third Determination}
\end{center}
\par The California Bearing Ratio (CBR) can be determined from the plots of the compression load results by plotting stress versus penetration depth. The CBR value of a soil sample is one characteristic used to determine its overall strength as it is correlated to the penetration resistance of a soil. In this experiment, the soil sample analyzed yielded a CBR of approximately 20.94 which corresponds to expected values of more coarse soil. Depending on the soil composition, a CBR value in the range of 15 to 50 can be indicative of well graded soil. Having a higher CBR value also correlates to the smaller required thickness of the subgrade level in pavement design. 
\newpage
\section{Classifications}
\par Soil generally consists of gravel, sand, silt, and clay. The predominant particle size usually dictates what the soil sample is called. Several organizations have developed particle-size classifications, as well as soil classification groups. Different soils with similar properties can be grouped together based on their engineering behavior. These Classification groups help to express the general characteristics of soils, without needing to develop a detailed description.  
\subsection{AASHTO}
\par The American Association of State Highway and Transportation Officials, AASHTO, classifies soils into seven major groups. There are two main classifications, granular and fine, which are further divided based on the percent of soil either passing through or retained above number 10, 40, and 200 sieves. These sieves separate the particle sizes for gravel, sand, silt, and clay. In addition to grain size, AASHTO classification also considers plasticity through the soils' liquid limit and plastic index. The grain size percentages, liquid limit, and plastic index for our soil sample is as follows:  
\[\% \text{ Passing No. 10}=49.34\%\hspace{4mm}\% \text{ Passing No. 40}=47.01\%\hspace{4mm}\% \text{ Passing No. 200}=11.74\%\]
\[LL=26.8\%\hspace{3mm}PI=-7.77\%\] 

\par These values are compared to the values present on the Classification of Highway Subgrade Materials chart to determine which group the soil best falls under. The AASHTO classification of the soil is $\boxed{\text{A-1-b (0)}}$. The group index for A-1-b classification is always 0. 
\subsection{USCS} 
\par The United Soil Classification System, USCS, classifies soils into two main groups: coarse-grained and fine-grained. Like AASHTO, for USCS, the percent of gravel, percent of sand, percent of silt and clay, liquid limit, and plasticity index are needed for proper classification. In addition to these values, the uniformity coefficient, and coefficient of gradation of the soil help with classification. Flow charts are used to determine the group symbol, group name, and description of a soil sample. Since 88.26\% was retained above the No. 200 sieve (greater than 50\%), the sample is coarse-grained. Since the percent finer for the No. 4 sieve is 47.01\% (less than 50\%), the sample falls under gravel. Because the plastic index is negative, it is effectively 0, indicating no plasticity. The plastic index of the A-line is compared to the actual value of the plastic index.  
\[\text{A-line PI} = 0.73(26.8-20) = 4.96\]
\par The Plastic Index is less than the A-line Plastic Index, therefore the sample falls below the A-line and can fall under the group symbols ML or OL. Using the grain size percentages, it can be determined that 50.26\% of the sample is gravel, and 38.00\% is sand. Because the percentage of sand is less than that of gravel, the sample can be labeled ML, gravelly silt with sand. More than 50\% of the sample is retained above both the No. 4 and No. 200 sieve. In addition, the Coefficient of gradation, $C_u$, is determined to be 254, and the uniformity coefficient, $C_c$, is 0.06. According to these values, the sample is GP-GM, poorly graded. The complete USCS Classification of the soil is $\boxed{\text{GP-GM, poorly graded gravel with silt and sand}}$.
\subsection{FAA} 
\par The Federal Aviation Administration, FAA, has a separate method for classifying soils that is designed for airport pavement design. FAA subclasses depend depends on soil classification, and whether the pavement is rigid or flexible. These subgrades are based on the performance of soil under different conditions of drainage and frost. `F' and `R' denote ``flexible'' or ``rigid'', and drainage implies that the soil under the pavement does not retain water. The designations for frost depends on geographical conditions. FAA classification utilized sieve analysis, plastic index, and liquid limits like AASHTO and USCS. 
\par For FAA, we look at the percent retained above the No. 10 sieve, 50.66\%. In addition, we look at the percent finer than No. 10 but retained above No. 40, 3.65\%, the percent finer than No. 40 but retained above No. 200, 41.25\%, and the percent finer than No. 200 sieve, 11.74\%. These values are compared to a chart defined by the FAA, so a subgroup could be determined. The group that this soil best falls under is E-5. In addition, because the percent passing through the No. 200 is less than 20\%, we can say that this soil has good drainage. The FAA Classification for this soil sample, therefore, is $\boxed{\text{E-5 (F1) for flexible pavement, no frost}}$.
\newpage
\subsection{Overall}
\par Overall soil descriptions are based on particle size and the percentage of the sample that falls within each classification of gravel, sand, silt, and clay. The AASHTO and USCS descriptions are shown below.
\begin{center}
    \begin{tabular}{|l|cc|}
        \hline
        \textbf{Grain Size Type} & \textbf{Percentage (AASHTO)} & \textbf{Percentage (USCS)} \\\hline 
        Gravel & 50.66\% & 50.26\%\\ 
        Sand & 37.60\% & 38\%\\ 
        Silt and Clay & 11.74\% & 11.74\%\\\hline
    \end{tabular}
    \vspace{3mm}
    \addcontentsline{lot}{table}{Table 10: AASHTO and USCS Descriptions}
    \emph{\\Table 6: AASHTO  and USCS Descriptions\\}
\end{center}
\par For both AASHTO and USCS, clay is described as any particles less than 0.002 mm. In our experiment, we did not use a sieve smaller than No. 200. Therefore, we cannot determine, with certainty, what percent of our soil sample is clay using particle size analysis. 
\par For FAA, silt and Clay are grouped into the same category. Note that the percentages for each type of grain size should add up to 100\% of the soil sample. Having a value greater than or less than 100\% is attributed to error in the completion of the sieve analysis. The FAA description is below.
\begin{center}
    \begin{tabular}{|l|r|}
        \hline
        \textbf{Grain Size Type} & \textbf{Percentage} \\\hline 
        Gravel & 50.66\% \\
        Coarse Sand & 3.65\% \\
        Fine Sand & 41.25\% \\
        Silt and Clay & 11.74\% \\\hline
    \end{tabular}
    \vspace{3mm}
    \addcontentsline{lot}{table}{Table 11: FAA Description}
    \emph{\\Table 7: FAA Description\\}
\end{center}
\newpage
\section{Appendix} 
\subsection{Data Sheets}
\begin{center}
\includegraphics*[scale=0.7]{ds1.pdf}
\newpage
\includegraphics*[scale=0.8]{ds2.pdf}
\includegraphics*[scale=0.8]{ds2_2.pdf}
\includegraphics*[scale=0.8]{ds3.pdf}
\includegraphics*[scale=0.8]{ds4.pdf}
\end{center}
\subsection{Sample Calculations}
\subsubsection{Specific Gravity Determination}
\noindent The \underline{specific gravity} is calculated using \emph{Equation \ref{gs}}.
\begin{equation}\label{gs}G_s=\frac{W_s}{W_{fw}+W_s-W_{fws}}\end{equation}
\[G_s=\frac{73.53\text{ g}}{680.02\text{ g}+73.53\text{ g}-726.24\text{ g}}=\boxed{2.72}\]
\subsubsection{Grain Size Distribution}
\noindent The \underline{soil fraction} is calculated using \emph{Equation \ref{ws}}. 
\begin{equation}\label{ws}W_\text{soil}=W_\text{container+soil}-W_\text{container}\end{equation}
\[W_\text{dry}=131.96\text{ g}-82.83\text{ g}=\boxed{49.13\text{ g}}\]
\noindent The \underline{weight of moisture} is calculated using \emph{Equation \ref{wm}}. 
\begin{equation}\label{wm}W_\text{Moisture}=W_\text{container+wet soil}-W_\text{container+dry soil}\end{equation} 
\[W_\text{moisture}=132.84\text{ g}-131.96\text{ g}=\boxed{0.88\text{ g}}\] 
\noindent The \underline{moisture content} is calculated using \emph{Equation \ref{w}}. 
\begin{equation}\label{w}w=\frac{W_\text{moist}}{W_\text{dry}}\times 100\end{equation} 
\[w=\frac{0.88\text{ g}}{49.13\text{ g}}\times 100=\boxed{1.79\%}\] 
\noindent The \underline{weight of dry fine fraction} is calculated using \emph{Equation \ref{wdf}}. 
\begin{equation}\label{wdf}W_\text{dry fines}=\frac{W_\text{wet fines}}{1+\frac{w}{100}}\end{equation} 
\[W_\text{dry fines}=\frac{546.37\text{ g}}{1+\frac{1.79\%}{100}}=\boxed{536.76\text{ g}}\] 
\noindent The \underline{weight of dry coarse and fine fraction} is calculated using \emph{Equation \ref{wcfd}}.
\begin{equation}\label{wcfd}W_\text{coarse + fine dry}=W_\text{dry coarse}+W_\text{fine dry}+(\text{-No. 10 Correction})\end{equation} 
\[W_\text{coarse + fine dry}=556.39\text{ g}+536.76\text{ g}+5.42\text{ g}=\boxed{ 1098.57\text{ g}}\]
\noindent The \underline{cumulative grams finer} is calculated using \emph{Equation \ref{cgf}}. 
\begin{equation}\label{cgf}\text{Cum. Grams Finer}=(\text{Grams Finer})_{i-1}-(\text{Grams Retained})_i\end{equation}
\[\text{Cum. Grams Finer (1 in.)}=1098.57\text{ g}-38.46\text{ g}=\boxed{1060.11\text{ g}}\]
\noindent The \underline{percent finer for the coarse and fine fraction} is calculated using \emph{Equation \ref{pfcf}}. 
\begin{equation}\label{pfcf}\%_\text{Finer}=\frac{\text{Cum. Grams Finer}}{W_\text{coarse + fine dry}}\times 100\end{equation}
\[\%_\text{Finer (1 in.)}=\frac{1060.11\text{ g}}{1098.57\text{ g}}\times 100=\boxed{96.5\%}\] 
\noindent The \underline{percent finer of the whole} is calculated using \emph{Equation \ref{pfow}}. 
\begin{equation}\label{pfow}\%_\text{Finer of Whole}=\frac{\%_\text{Finer No. 10}}{100}\times \frac{pf_\text{max}}{100}\times 100\end{equation}
\[\%_\text{Finer of Whole (No. 16)}=\frac{99.73}{100}\times\frac{49.74}{100}\times 100=\boxed{49.6\%}\] 
\noindent \underline{$H_r'$} is calculated using \emph{Equation \ref{hrr}}. 
\begin{equation}\label{hrr}H_r'=H+\frac{\text{Bulb Height}}{2}\end{equation}
\[H_r'=2.7+\frac{14.4\text{ cm}}{2}=\boxed{9.9\text{ cm}}\] 
\noindent \underline{$H_r$} is calculated using \emph{Equation \ref{hr}}. 
\begin{equation}\label{hr}H_r=H_r'-\frac{\text{Volume}}{2\times\text{ Area}}\end{equation}  
\[H_r=9.9-\frac{68\text{ cm}^3}{2\times 27.78\text{ cm}^2}=\boxed{8.68\text{ cm}}\]
\noindent \underline{$R_H''$} is calculated using \emph{Equation \ref{Rhh}}. 
\begin{equation}\label{Rhh}R_H'+c_m\end{equation}
\[R_H''=1.013+0.0002=\boxed{1.0132}\]
\noindent \underline{$R_H$} is calculated using \emph{Equation \ref{Rh}}. 
\begin{equation}\label{Rh}R_H''\pm c_d\end{equation}
\[R_H''=1.013\pm 0.0018=\boxed{1.0114}\]
\noindent The \underline{distance} is calculated using \emph{Equation \ref{dist}}. 
\[D^2=\frac{18\mu\times v}{\gamma_w-(\gamma_w+c_d)}\]
\begin{equation}\label{dist}d=D^2\times t\times \frac{G_s-(\gamma_w+c_d)}{18\times \mu}\end{equation}
\[d=\frac{\left(\frac{0.069}{10}\right)^2\times 30\times (2.8-(0.9982+0.0018))}{18\times(1029\times 10^{-8})}=\boxed{13.99\text{ cm}}\]
\noindent The \underline{uniformity coefficient} is calculated using \emph{Equation \ref{cu}}. 
\begin{equation}\label{cu}C_u=\frac{D_{60}}{D_{10}}\end{equation} 
\[C_u=\frac{10.5}{0.51}=\boxed{205.88}\]
\noindent The \underline{coefficient of gradation} is calculated using \emph{Equation \ref{cc}}. 
\begin{equation}\label{cc}C_c=\frac{D_{30}^2}{D_{60}\times D_{10}}\end{equation} 
\[C_c=\frac{0.2^2}{10.5\times 0.51}=\boxed{0.075}\] 
\noindent The \underline {percent soil type} is calculated two ways one using \emph{Equation \ref{pst1}} and the alternative way using \emph{Equation \ref{pst2}}. 
\begin{equation}\label{pst1}\%_\text{Soil Type}=\%_\text{Greater}-\%_\text{Finer}\end{equation}
\[\%_\text{Medium Gravel}=100\%-54.94\%=\boxed{45.06\%}\] 
\begin{equation}\label{pst2}\%_\text{Soil Type}=\sum[\%_\text{retained on each sieve}]\end{equation}
\[\%_\text{Fine Sand}=1.03\%+6.29\%+1.83\%=\boxed{9.15\%}\]
\subsubsection{Atterberg Limits}
\noindent The \underline{moisture content} is calculated using the equation below, with $W_m$ being the weight of the moisture, $W_d$ being the weight of the dry soil, and $W_c$ being the weight of the container, as seen in \emph{Equation \ref{w2}}. 
\begin{equation}\label{w2} w=\frac{W_m}{W_d-W_c}\times 100 \end{equation}
\[w=\frac{2.99\%}{85.92\%-74.61\%}\times 100 = \boxed{26.43\%}\]
\noindent The \underline{Liquid Limit} is calculated by taking the sum of the water content and dividing it by the number of trials, as seen in \emph{Equation \ref{ll}}.
\begin{equation}\label{ll} LL=\frac{\sum w}{w_n} \end{equation}
\[LL=\frac{24.44\%+29.58\%+27.92\%+24.93\%+25.35\%}{5}=\boxed{26.84\%}\]
\noindent The \underline{Plastic Limit} is calculated similarly, as seen in \emph{Equation \ref{pl}}.
\begin{equation}\label{pl} PL=\frac{\sum w}{w_n} \end{equation}
\[PL=\frac{23.54\%+16.28\%+64.05\%}{3}=\boxed{34.62\%}\]
\noindent The \underline{Plastic Index} is calculated using \emph{Equation \ref{pi}}. 
\begin{equation}\label{pi}PI=LL-PL\end{equation} 
\[PI=26.84\%-34.62\%=\boxed{-7.78\%}\]
\noindent The \underline{Flow Index} is calculated using \emph{Equation \ref{fi}}. $w$ and $N$ represent the moisture content and number of blows respectively. 
\begin{equation}\label{fi}FI=\frac{w_1-w_2}{\log\left(\frac{N_2}{N_1}\right)}\end{equation}  
\[FI=\frac{26.43\%-29.58\%}{\log\left(\frac{20}{13}\right)}=\boxed{17.65\%}\]
\noindent The \underline{Liquidity Index} is calculated using \emph{Equation \ref{li}}. 
\begin{equation}\label{li}LI=\frac{w-PL}{PI}\end{equation} 
\[LI=\frac{26.43\%-34.62\%}{-7.78\%}=\boxed{1.05}\]
\noindent The \underline{Activity} is calculated using \emph{Equation \ref{a}}.
\begin{equation}\label{a}A=\frac{PI}{\%_\text{Clay}}\end{equation} 
\[A=\frac{0\%}{0\%}=\boxed{\text{Undefined}}\]
\subsubsection{Compaction Test}
\subsection{Soil Classifications}
\begin{center}
\includegraphics*[scale=0.5]{fig1.png}
\addcontentsline{lof}{figure}{Figure 10: FAA Classification Table}
\emph{\\Figure 10: FAA Classification Table\\}
\vspace{5mm}
\includegraphics*[scale=0.5]{fig2.png}
\addcontentsline{lof}{figure}{Figure 11: Plasticity Chart}
\emph{\\Figure 11: Plasticity Chart\\}
\vspace{10mm}
\includegraphics*[scale=0.5]{fig3.png}
\addcontentsline{lof}{figure}{Figure 12: AASHTO Classification Table}
\emph{\\Figure 12: AASHTO Classification Table\\}
\includegraphics*[scale=0.8]{fig4.png}
\addcontentsline{lof}{figure}{Figure 13: USCS Classification Table}
\emph{\\Figure 13: USCS Classification Table\\}
\includegraphics*[scale=0.7]{fig5.png}
\addcontentsline{lof}{figure}{Figure 14: Flow Chart for Group Names}
\emph{\\Figure 14: Flow Chart for Group Names\\}
\includegraphics*[scale=0.8]{fig6.png}
\addcontentsline{lof}{figure}{Figure 15: Flow Chart for Group Names of Inorganic Silty and Clayey Soils}
\emph{\\Figure 15: Flow Chart for Group Names of Inorganic Silty and Clayey Soils\\}
\end{center}
\newpage

\end{document}